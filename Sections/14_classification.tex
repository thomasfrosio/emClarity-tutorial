\section{Classification} \label{sec:classification}

\subsection{Objectives}

The heterogeneity of the data-set can be analysed by comparing individual particles with the current subtomogram average. To correct for differences in sampling between the reference and an individual particle, the current subtomogram average is distorted by the particle's 3D-sampling function it is being compared to. This effectively estimates what the average particle should look like at that subtomogram position, allowing to only compare meaningful differences. The dimensionality of these differences is reduced by principal component analysis, using singular-value decomposition (SVD). Features of a given length scale (e.g. $\alpha$-helices, small protein domains, etc.) can be focused on and considered simultaneously by band-pass filtering the reconstructions and computing the SVD for each length scale. The singular vectors describing the greatest variance for each length scale are then concatenated into feature vectors and clustered with $k$-means or other clustering algorithms.


\subsection{Parameters}

% Parameters for pca
\renewcommand{\arraystretch}{1.2}
\begin{longtable}[l]{| l || p{110mm} |}
\captionsetup{labelfont=bf}
\caption[\code{pca} parameters]{\code{pca} parameters. Your parameter file should have the following parameters. \textcolor{myred}{\textbf{*}} indicates the required parameters, \textcolor{blue}{\textbf{*}} indicates expert parameters. Expert parameters should not be changed except if you know what you are doing. The other parameters are optional.}\\

\hline
\multicolumn{2}{|c|}{\textbf{Sampling}}\\
\hline

-- \code{PIXEL\_SIZE}\textcolor{myred}{\textbf{*}} & Pixel size in meter per pixel (e.g. 1.8e-10). Must match the header of the stacks in \code{fixedStacks/*.fixed}.\\
-- \code{SuperResolution}\textcolor{myred}{\textbf{*}} & Whether or not the \code{fixedStacks/*.fixed} are super-sampled. Not that this should be the same value you used for \code{ctf estimate} in section \ref{sec:defocus_estimate}.\\
-- \code{Cls\_samplingRate}\textcolor{myred}{\textbf{*}} & Current binning factor (1 means no binning). The sub-region tomograms at this given binning must be already reconstructed in the \code{cache} directory. If they aren't, you'll need to run \code{ctf 3d} before running this step.\\

-- \code{Ali\_samplingRate}\textcolor{myred}{\textbf{*}} & Binning factor (1 means no binning) of the half-maps of the current cycle. If this is different from \code{Cls\_samplingRate}, the reconstructions will be resampled to match \code{Cls\_samplingRate}.\\


\hline
\multicolumn{2}{|c|}{\textbf{Masks}}\\
\hline

-- \code{Ali\_mType}\textcolor{myred}{\textbf{*}} & Type of mask used for the reconstruction; ``cylinder'', ``sphere'', ``rectangle''.\\
-- \code{Cls\_mType}\textcolor{myred}{\textbf{*}} & Type of mask to use for the PCA; ``cylinder'', ``sphere'', ``rectangle''.\\

-- \code{Ali\_Radius}\textcolor{myred}{\textbf{*}} & [$x,\ y,\ z$] mask radius, in \r{A}, used for the reconstruction.\\
-- \code{Cls\_Radius}\textcolor{myred}{\textbf{*}} & [$x,\ y,\ z$] mask radius, in \r{A} to use for the PCA.\\

-- \code{Ali\_mCenter}\textcolor{myred}{\textbf{*}} & [$x,\ y,\ z$] shifts, in \r{A}, used for the reconstruction. These are relative to the center of the reconstruction. Positive shifts translate the \code{Ali\_mType} mask to the right of the axis.\\
-- \code{Cls\_mCenter}\textcolor{myred}{\textbf{*}} & [$x,\ y,\ z$] shifts, in \r{A} to use for the PCA. These are relative to the center of the reconstruction. Positive shifts translate the \code{Ali\_mType} mask to the right of the axis.\\

-- \code{flgPcaShapeMask} & Calculate and apply a molecular to the difference maps. This molecular mask is calculated using the combined reference. Default=true.\\

-- \code{test\_updated\_bandpass} & By default (0/false), low-pass filters are used to calculate the length scales. If true, use band-pass filters. See section \ref{sec:algo:classification:resolution_bands} for more details.\\

\hline
\multicolumn{2}{|c|}{\textbf{PCA}}\\
\hline

-- \code{pcaScaleSpace}\textcolor{myred}{\textbf{*}} & Length scales, i.e. resolution bands, in \r{A}. If this is a vector, the PCA will be performed for each length scales and you will need to select the principal axes for each length scale.\\

-- \code{Pca\_randSubset}\textcolor{myred}{\textbf{*}} & For very large data sets (tens of thousands) the principal axes describing the important variation can be obtained with a subset of the data, which saves a lot of computation. If 0, use the entire dataset, otherwise specify the number of particle that should be randomly selected to perform the decomposition. Usually 25\% or at least 3000-4000 is a good way to go.\\

-- \code{Pca\_maxEigs}\textcolor{myred}{\textbf{*}} & Most of the variance is usually explained within the first 20 to 30 directions, so it is usually not useful to save all of the directions. Use this parameter to select the number of principal directions to save.\\

\hline
\multicolumn{2}{|c|}{\textbf{Clustering}}\\
\hline

-- \code{Pca\_coeffs}\textcolor{myred}{\textbf{*}} & The selected principal axes, for each length scale. Each length scale is a row. You can select as many (or few) from any resolution as you want, but the number of entries in each row must be constant. Use zeros to fill empty places.\\

-- \code{Pca\_clusters}\textcolor{myred}{\textbf{*}} & The number of clusters to find. If this is a vector, the clustering will be calculated for each registered value.\\

-- \code{Pca\_distMeasure} & This corresponds to the \code{Distance} entry of the \code{kmeans} function of {\MATLAB}. By default, the squared Euclidean distance metric, i.e. each centroid is the mean of the points in that cluster.\\

-- \code{Pca\_nReplicates} & This corresponds to the \code{Replicates} entry of the \code{kmeans} function of {\MATLAB}. By default, the number of replicates is set to 128, i.e. the number of times to repeat clustering using new initial cluster centroid positions.\\

\hline
\multicolumn{2}{|c|}{\textbf{Others}}\\
\hline

-- \code{PcaGpuPull}\textcolor{myred}{\textbf{*}} & The decomposition is calculated on the CPU, but the difference maps are calculated on the GPU, which is much more efficient. This parameters controls how many difference maps should be held on the GPU at any given type.\\

-- \code{flgClassify}\textcolor{myred}{\textbf{*}} & Specify that the this cycle is a classification cycle. Must be set to 1/true.\\

-- \code{subTomoMeta}\textcolor{myred}{\textbf{*}} & Project name. At this step, {\emClarity} excepts to find the metadata \code{subTomoMeta}.mat in the project directory.\\

-- \code{flgCutOutVolumes} & Whether or not each transformed particle (rotated and sifted) used to calculate the difference maps should be saved to \code{cache} directory. Note that the subtomogram have an extra padding of 20 pixel. This makes the pre-processing for the PCA much slower if activated. Default=0.\\

-- \code{scaleCalcSize}\textcolor{blue}{\textbf{*}} & Scale the box size used to calculate the difference maps by this number. Default=1.5.\\

-- \code{use\_v2\_SF3D} & Whether or not the new per-particle sampling function procedure should be used, as opposed to the older ``grouped'' sampling functions. This is the default since {\emClarity} 1.5.1.0. Default=1.\\

\hline
\end{longtable}








% SuperResolution
% Pca_symMask, only supports cylinder for some reason, so don't talk about it.
% Pca_flattenEigs, 0


\subsection{PCA}

\subsubsection{Run}

Before running the PCA, you have to start a new cycle by running the following command:
\begin{lstlisting}
>> emClarity avg <param.m> <cycle_nb> RawAlignment
\end{lstlisting}
Then, you can run the PCA on the entire dataset:
\begin{lstlisting}
>> emClarity pca <param.m> <cycle_nb> 0
\end{lstlisting}
For very large data sets, i.e. tens of thousands particles, the principal directions describing the important variation can be obtained with a subset of the data, which saves a lot of computation. In this case, use the \code{Pca\_randSubset} parameter and run the same command shown above. Then, to calculate the principal components of the rest of the dataset, run this second command:
\begin{lstlisting}
>> emClarity pca <param.m> <cycle_nb> 1
\end{lstlisting}

\subsubsection{Outputs}

The goal now is to select, for each length scale, the principal directions that are going to be used to reproject the data onto. There are a few files that are here to help you decide, all of which are introduced in section \ref{sec:algo:classification:SVD}.

For example, let's say you have 3 length scales and want to select the principal directions 3 to 5 for the first scale, 4, 5, 8, 9 for the second scale, and 3 and 5 for the last scale, you would have to write in your parameter file:

\code{Pca\_coeffs=[ 3:5,0 ; 4,5,8,9 ; 3,5,0,0 ]}, or equivalently\\
\code{Pca\_coeffs=[ 3,4,5,0 ; 4,5,8,9 ; 3,5,0,0 ]}

In some cases, it is not obvious which principal directions to choose and it might require some trials before successfully clustering the dataset into meaningful clusters. Moreover, it might not be clear if the classification was successful or not until the reconstruction step.

\subsection{Clustering}

\subsubsection{Run}

Once that you have selected the principal axes, for each length scale, we can select the corresponding principal components and cluster them.

You can specify how many clusters you want with the \code{Pca\_clusters} parameters. For example, say you want to run 3 clustering, one with 2 clusters, one with 4 and one with 6:

\code{Pca\_clusters=[2,4,6]}

Then, run the following command:
\begin{lstlisting}
>> emClarity cluster <param.m> <cycle_nb>
\end{lstlisting}

\subsubsection{Outputs} \label{sec:classification:clustering:outputs}

For each requested number of clusters (\code{Pca\_clusters}), the class populations are printed into a text file in the
project directory called \code{<projectName>\_cycleXXX\_ClassIDX.txt}.


\subsection{Reconstruction}

To create a montage of your class averages, we need to call the \code{avg} procedure. As such, your parameter file should contains the parameters necessary to run \code{avg} (table \ref{param:avg}), plus an additional 3 parameters. Indeed, you have to specify which clustering to use. Say you ran 3 clustering, one of them with 6 clusters, like in the previous example. To reconstruct the subtomogram average of these 6 classes, you have to specify the following parameters in your parameter file:

\code{Cls\_className=6}\\
\code{Cls\_classes\_odd=[1:6,1.*ones(1,6)]}\\
\code{Cls\_classes\_eve=[1:6,1.*ones(1,6)]}, or equivalently\\
\code{Cls\_classes\_eve=[1:6, 1,1,1,1,1,1]}

Each entry in the second row (everything after \code{1:6,}) is a CX symmetry, which may not be the same for each class. For example, you first ran the reconstruction without symmetry and realize that you have a C3 symmetry for all the classes, except in the last 2 classes where there is a C6 symmetry, you could then rerun the reconstruction with the following parameters:

\code{Cls\_classes\_odd=[1:6, 3,3,3,3,6,6]}\\
\code{Cls\_classes\_eve=[1:6, 3,3,3,3,6,6]}

\begin{note}If you use the new \code{symmetry} parameter, the second row is not used but is still required for compatibility reasons.\end{note}

To call the \code{avg} function, run the following command:
\begin{lstlisting}
>> emClarity avg <param.m> <cycle_nb> Cluster_cls
\end{lstlisting}
This can be done for each of your clustering specified in the \code{Pca\_clusters} parameter.


\subsection{Select the classes to use as references}

At this point, there are two possibilities. If any of the clustering was successful and you would like to cancel the classification and switch back to averaging and alignment, you could either use and rename the metadata back-up from the previous cycle and continue as nothing happened, or you could skip the current cycle by running the following command:
\begin{lstlisting}
>> emClarity skip <param.m> <cycle_nb>
\end{lstlisting}

On the other hand, if you have identified a good clustering, with some classes you would like to keep and some classes you would like to exclude, you will have to:
\begin{enumerate}
    \item Make sure the last reconstruction was done using the clustering you want to use. For example, if you ran the clustering with \code{Pca\_clusters=[2,4,6]}, then reconstructed the montage for all of these clustering, starting from 2 to 6, but want to use the second clustering with 4 classes, you would have to rerun the reconstruction of the second clustering before continuing.
    % I think that's a bug, at least in my experience this is a necessary step. I didn't pay attention to this when I was locking at the code tbh...
    
    \item Save a copy of your metadata. This is useful for two reasons. First, the next step will remove some particles from further consideration, so if something goes wrong, it's better to keep a copy of the current state of the metadata. Second, if you want to split your dataset in multiple parts to process some classes independently from the others, the easiest way is to use this copy to repeat the next steps for each (group of) classes you want to analyze.

    \item \textbf{Select the classes you want to exclude}: Open your class montage in \code{3dmod}, in ``Model'' mode and create a new model point for each class you want to delete. Save the model file, for example as \code{classes2ignore.mod}.
    
    \item \textbf{Exclude these classes}: Run the following command to exclude them; the particles that belongs to this/these class/classes will be tagged with a \code{-9999} in the metadata.
\begin{lstlisting}
>> emClarity geometry <param.m> <cycle_nb> Cluster_cls RemoveClasses <classes2ignore.mod> STD
\end{lstlisting}
    {\emClarity} will tell you how many particles were removed and how many are left. This should correspond exactly to the class populations from the clustering (section \ref{sec:classification:clustering:outputs}). If it doesn't, stop and make sure you followed the instructions from the first step. Replace the metadata with your copy from step 2 and try again.

    \item \textbf{End this cycle of classification}: Finally, you will need to ``skip'' the currently disabled class average alignment, and
    update the metadata so you may proceed to the next cycle of averaging/alignment.
\begin{lstlisting}
>> emClarity skip <param.m> <cycle_nb>
\end{lstlisting}
\end{enumerate}

At this point, you can then turn off classification in your next parameter file (\code{flgClassify=0}) and proceed to the next cycle.