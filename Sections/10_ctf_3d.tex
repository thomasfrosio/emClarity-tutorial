\section{Reconstruct the tomograms} \label{sec:ctf_3d}

\subsection{Objectives}

The objective here is to reconstruct the tomograms that will be later used to extract the particles and calculate the references (i.e the half-maps). For high-resolution sub-tomogram averaging, we must account for a defocus gradient perpendicular to the til-axis, as well as a gradient through the thickness of the sample due to Ewald sphere curvature. The former is handled by multiplying the Fourier transform of each projection by their measured CTFs. In addition to correcting for contrast inversions, multiplication (rather than phase-flipping) also helps to supress noise near the CTF zeros. Previous approaches multiplied the CTF on thin strips or small tiles, which can lead to significant aliasing (see \cite{Tegunov2020} Figure S2 for a nice explanation). In {\emClarity}, the full tilt image is multiplied by the CTF for a given defocus, inverse Fourier transformed and then pixels corresponding to that defocus are extracted. This is repeated over the range of defoci in the image, thus avoiding any aliasing. Each view is also weighted according to the cumulative electron dose as described in \cite{exposure_grant_2015}. {\emClarity} then uses the {\IMOD} program {\tilt} to reconstruct the tomograms, preserving the full context of the tomogram while simultaneously considering local movement and anisotropic magnification due to microlensing from charge accumulation on the specimen \cite{MASTRONARDE2017102}.

\subsection{Parameters}
% Parameters for ctf 3d
\renewcommand{\arraystretch}{1.2}
\begin{longtable}[l]{| l || p{115.5mm} |}
\captionsetup{labelfont=bf}
\caption[\code{ctf 3d} parameters]{\code{ctf 3d} parameters. Your parameter file should have the following parameters.\\ \textcolor{myred}{\textbf{*}} indicates the required parameters, \textcolor{blue}{\textbf{*}} indicates expert parameters. Expert parameters should not be changed except if you know what you are doing. The other parameters are optional.}\\


%% Parameters:
% flgDampenAliasedFrequencies (default=0)

\hline
\multicolumn{2}{|c|}{\textbf{Microscope settings}}\\
\hline

-- \code{PIXEL\_SIZE}\textcolor{myred}{\textbf{*}} & Pixel size in meter per pixel (e.g. 1.8e-10). Must match the header of the stacks in \code{fixedStacks/*.fixed}.\\
-- \code{SuperResolution}\textcolor{myred}{\textbf{*}} & Whether or not the \code{fixedStacks/*.fixed} are super-sampled. Not that this should be the same value you used for \code{ctf estimate} in section \ref{sec:defocus_estimate}.\\

-- \code{Ali\_samplingRate}\textcolor{myred}{\textbf{*}} & Binning factor (1 means no binning) of the output reconstruction.\\

% CTF correction
\hline
\multicolumn{2}{|c|}{\textbf{CTF correction}}\\
\hline

-- \code{useSurfaceFit}\textcolor{blue}{\textbf{*}} & Whether or not the spatial model should be calculated as a function of $x,\ y$ coordinates. If 0, the spatial model is a plane (constant center-of-mass). See section \ref{sec:algo:ctf_3d:spatial_model} for more details.\\

-- \code{flg2dCTF}\textcolor{blue}{\textbf{*}} & Whether or not the CTF correction should correct for the defocus gradients along the electron beam (thickness of the specimen). If 1, only one $z$ section is used. See section \ref{sec:algo:ctf_3d:defocus_step} for more details.\\

\hline
\multicolumn{2}{|c|}{\textbf{Others}}\\
\hline

-- \code{erase\_beads\_after\_ctf} & Whether or not the fiducial beads should be removed before or after CTF multiplication. Default=\code{false}.\\

-- \code{applyExposureFilter} & Whether or not the exposure filter should be applied. If you turn it off, make sure it is turned-off during subtomogram averaging as well.\\

-- \code{super\_sample}\textcolor{blue}{\textbf{*}} & Compute the back projection in a slice larger by the given integer factor (max=8) in each dimension, by interpolating the projection data at smaller intervals ("super-sampling"). This corresponds to the \code{SuperSampleFactor} entry from {\tilt}. Default=\code{0}.\\

-- \code{expand\_lines}\textcolor{blue}{\textbf{*}} & If \code{super\_sample} is greater than 0, expand projection lines by Fourier padding (sync interpolation) when super-sampling, which will preserve higher frequencies better but increase memory needed. This corresponds to the \code{ExpandInputLines} entry from {\tilt}.Default=\code{0}\\

\hline
\end{longtable}


\newpage

\subsection{Run}

To run it, you just need to specify your parameter file and {\emClarity} will look at the project metadata to extract the current alignment cycle. If your \code{cache} directory already contains reconstructions at this binning (\code{Ali\_samplingRate}), {\emClarity} will not override them. As such, you should remove any previous reconstructions at that binning from your \code{cache}.
\begin{lstlisting}
>> emClarity ctf 3d <param.m>
\end{lstlisting}

\subsection{Outputs}

The binned tilt-series and CTF-corrected reconstruction are saved in the \code{cache} directory.