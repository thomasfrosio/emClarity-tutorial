\section{Algorithms} \label{sec:algo}

\subsection{Naming conventions} \label{sec:algo:naming_conventions}

There is a lot of things to cover and it is often easier to use abbreviations (CTF, FSC, CCC, etc.) and symbols to refer to something.

% \renewcommand{\arraystretch}{1.2}
\begin{longtable}[c]{| l || p{120mm} |}
\captionsetup{labelfont=bf}
\caption{Symbols often used} \label{tab:symbols}\\
% I need to check the diff between first and last column. It is useful when removing images.

\hline
\textbf{Symbol} & \textbf{Description}\\ \hhline{|=#=|}
$\bm{S}$ & A reference, i.e. a subtomogram average or a template.\\ \hline
$\bm{s}$ & A particle in 3D, i.e. a subtomogram.\\ \hhline{|=#=|}

$\bm{V}$ & A tomogram. This usually refers to the full tomogram or a sub-region tomogram.\\ \hline
$\bm{I}$ & An image. This can be an entire image, a strip or a tile.\\ \hhline{|=#=|}

$\bm{W}$ & A weight in Fourier space. This can be a low- high- band-pass filter, a total 3D sampling function, 1D or 2D CTFs, an exposure or "B-factor" filter, etc. 3D sampling functions are referred as $\bm{w}$.\\ \hline
$\bm{M}$ & Real space masks. This can be any shape mask, molecular masks, evaluation masks, etc.\\ \hhline{|=#=|}

$\bm{\mathrm{z}}$ & a defocus value\\ \hline
$\bm{\Delta \mathrm{z}}$ & a defocus shift. $\bm{\Delta \mathrm{z}}_{ast}$ is the astigmatic shift.\\ \hline
$\bm{\phi}$ & the azimuthal angle. $\bm{\phi}_{ast}$ is the astigmatic angle.\\ \hhline{|=#=|}

$\bm{R}$ & a rotation matrix\\ \hline
$\bm{T}$ & a translation\\ \hline
$\bm{\alpha}$ & a tilt angle\\ \hline

\end{longtable}

Indexes are subscripts, e.g. the $p^{th}$ subtomogram is referred as $\bm{s}_p$. This works with multiple indexes, e.g. the $p^{th}$ subtomogram rotated by the $r^{th}$ rotation is referred as $\bm{s}_{p,r}$. Labels are subscripts as well, e.g. if we want to specify that the subtomograms are in the reference frame, we would write $\bm{s}_{ref}$. On the other hand, if the symbol is labeled \emph{and} an index is needed, the labeled symbol is placed between square brackets. For instance, if we want to specify that the $p^{th}$ subtomogram is in the reference frame, we would write ${[\bm{s}_{ref}]}_p$.


\subsection{Euler angles conventions} \label{sec:algo:euler_conventions}

The $\phi, \theta, \psi$ Euler angles used by {\emClarity} describe a $z\text{-}x\text{-}z$ active intrinsic rotations of the particles coordinate system. That is to say, to switch the particles from the microscope frame to the reference frame, the basis vectors of the subtomograms are rotated (positive anti-clockwise) around $z$, the new $x$, and the new $z$ axis.

The microscope frame defines the coordinate system of the microscope, where the electron beam is the $z$ axis. When the subtomograms are extracted from their tomogram, they are in the microscope frame. The reference frame is the coordinate system attached to the reconstruction, i.e. the subtomogram average and is usually set during the particle picking.


\subsection{Linear transformations in Fourier space}

Linear transformations are often applied in Fourier space directly. It might be useful to write down a few useful properties of the Fourier transforms.

\begin{itemize}
    \item \textbf{Shift}: Shifting an image in real space is equivalent to applying a complex phase shift to its frequency spectrum, such as:
    \begin{equation}
        g(x,y) = f(x\bm{-a},y\bm{-b})\ \xleftrightarrow{\mathcal{F}}\ G(u,v) = F(u,v) \times \bm{ e^{ -2\pi i \left( \frac{au}{N}\ +\ \frac{bv}{M} \right) } }
    \end{equation}
    where $N$ and $M$ are the number of $u$ and $v$ frequencies, respectively. The complex phase shift is just a complex number on the unit circle, so the magnitude of the Fourier transform is unchanged, so $\abs{G(u,v)} = \abs{F(u,v)}$.
            
    \item \textbf{Magnification}: Magnifying an image by a factor $a$ is equivalent to magnifying its frequency spectrum by $1/a$, such as:
    \begin{equation}
        g(x,y) = f(\bm{a}x, \bm{b}y)\ \xleftrightarrow{\mathcal{F}}\ G(u,v) = \bm{\frac{1}{\abs{ab}}} \times F\left(\frac{u}{\bm{a}},\frac{v}{\bm{b}}\right)
    \end{equation}
            
    \item \textbf{Rotation}: Rotating an image by an angle $\Theta$ in real space is the same as rotating its frequency spectrum by the same angle $\Theta$.
\end{itemize}


\newpage

\subsection{Tilt-series alignment} \label{sec:algo:tilt_series_alignment}

\subsubsection{Pre-processing}

\subsubsection{Patch tracking}

\subsubsection{Refining on beads}

%% WORFLOW:

% 1) If skip view, create a copy of the series with these images removed.
% 2) If the image rotation - 180 > atand(nY./nX):
%       - rotate the series by 90deg with newstack.
% 3) Pre-process the tilt-series:
%       - a) ifft(fft(img)*nyquist)
%       - b) median filter; kernel=3 if pixel>2 else 2
%       - c) bandpass * fft(img); LOW_RES_CUTOFF=800, RESOLUTION_CUTOFF=autoAli_max_resolution
%       - d) resample img with the image rotation
% 4) autoAlign:
%       - a) for each binnning[binHigh:binInc:binLow]: (binInc = ceil((binHigh - binLow) / 3) ) AND for each iteration with this binning[autoAli_iterations_per_bin]:
%               - If first iteration, run tiltxcorr using the resampled tilt-series, without patch mode, just global.
%               - If >1 iteration, combine transformation (img rota + last iteration), apply them with newstack, run tiltxcorr in patch mode to get the fiducials and use them for tiltalign. TiltOption 0
%                   autoAli_n_iters_no_rotation is setting the number of iteration before activating the local alignments for tiltalign.
% 5) If rotate 90, rotate again but this time on the real .fixed (before it was preprocessed for alignment in tmp)

% 6) If refine on beads, for [15:-2:5]:
%       - apply last transformation with newstack, reconstruct with tilt(thick=3000), findbeads3d, reproject with tilt (.erase), imodtrans to set .erase relative to the original series,
%         then beadtrack to have a first estimate
%       - refine the bead positions with fitBeads:
%           - a) calculate the avg beads, center and standardize. Apply R-weight in Fourier space to keep low frequencies away (look at the edges!) before taking the average.
%           - b) Fit gaussian mixture (3 components) to this average and use this model 
%           - c) reconstruct this reference bead, pad to img size, and CC with img_derivative (R-weighted).
%           - d) go the original position of the beads, get tile, apply peakMask and update coords using the COM of the tile.
%       - Finally, run tiltalign using this current fiducial model. if (nBeads < 5) then return. if (nBeads < 11), no local alignment. TiltOption 5 and LocalTiltOption 0.
%
% OR
%
% 7) If no refine on beads: apply the final transformation to with newstack at bin 15. Then reconstruct with tilt at thickness 3000, then findbeads3d and reproject with tilt again.
%    The is used to get the .erase and _3dfind.ali.
%       

\newpage

\subsection{Defocus estimate} \label{sec:algo:defocus_estimate}

The defocus estimate is used at almost every step of the workflow to compute the theoretical CTFs. This procedure, called by \code{ctf estimate} is divided into 3 main steps. First, the raw tilt-series is transformed to compute the aligned tilt-series. Second, the average defocus of the entire aligned stack is estimated. Finally, the defocus is refined for each view.

\subsubsection{Transform the tilt-series} \label{sec:algo:defocus_estimate:transform}

\begin{enumerate}
    \item \textbf{Pre-processing}: For each view of the input stack, ``hot'' pixels, i.e. pixels 6 standard deviations away from the mean of the view, are replace by random scaled values. Moreover, frequencies after Nyquist are removed by low-pass filtering.

    \item \textbf{Transformation}: Each view of the pre-processed stack is transformed using the alignment files in \code{fixedStacks}, which effectively align the tilt-axis to the center of the images and parallel to the $y$ axis.
    \begin{enumerate}
        \item Load \code{fixedStacks/<prefix>.xf}, decompose the rotation matrix into magnification and rotation $\bm{R}$, and get the $\bm{T}_x$ and $\bm{T}_y$ shifts.
        \item Pad the view with its mean by a factor of 2, compute the forward Fourier transform and apply the magnification, rotation and shifts to the transform with bilinear interpolation. The spectrum is then Fourier cropped by a factor of 2 and inverse Fourier transformed to switch it back to real space. Increasing the sampling for the transformation reduces noise due to aliasing. Non-sampled regions are set to the mean, which is set to 0.
        
        \item Apply the same transformation to a mask of ``ones'' of the same size as the view. This mask is used to track down non-sampled regions of the view. The stack of masks (1 mask per view) is saved as \code{aliStacks/<prefix>\_ali1.samplingMask} and will be used later in the workflow.
    \end{enumerate}
    
    \item \textbf{Erase the beads}: If \code{fixedStacks/<prefix>.erase} exists, spherical masks of diameter $1.2\times\code{beadDiameter}$ and positioned at the coordinates specified by this file, are used to replace the beads by random scaled values. The coordinates are applied directly on the aligned stack, not on the raw stack.

    \item \textbf{Post-processing}:
    \begin{enumerate}
        \item In case ``hot'' pixels were introduced during transformation, these pixels are removed as in step \textbf{1} from the sampled regions (un-sampled pixels are delimited by the sampling mask from step \textbf{2.c}).
        % I'm not sure this ever happens. Maybe it's for something else?

        \item To keep track of the increasing thickness with the tilt angle, the views $i$ are divided by their fraction of inelastic scattering events, $\bm{f}_{inelastic}$ defined as:
        
        \begin{equation} \label{eq:f_inelatic}
        {[\bm{f}_{inelastic}]}_i = {\exp\left( \frac{-\bm{t}}{\cos(\bm{\alpha}_{i}) \times \bm{t}_{inelastic}} \right)}\ /\ {\exp\left( \dfrac{-\bm{t}}{\cos(\bm{\alpha}_{min}) \times \bm{t}_{inelastic}} \right)}
        \end{equation}

        where $\bm{t}$ is the thickness estimate of the tomogram, which is set to 75nm, $\bm{\alpha}_i$ is the tilt-angle of the $i^{th}$ view, $\bm{\alpha}_{min}$ is the lowest tilt-angle for the tilt-series and $\bm{t}_{inelastic}$ is the mean-free path of the inelastic scattering events, which is set to 400nm. 
        % The thickness estimate can be later refined during \code{tomoCPR} (section \ref{sec:tomoCPR}). => I thought so but actually no
        \item Similarly, if a Saxton/cosine dose-scheme is used, the views $i$ are divided by $1/\cos(\bm{\alpha}_i)$.
    \end{enumerate}

    \item \textbf{Save tilt-series}: Save the aligned, fraction weighted, bead-erased tilt-series as \code{aliStacks/<prefix>\_ali1.fixed}. These stacks are used by \code{templateSearch} (section \ref{sec:picking}) and \code{ctf 3d} (section \ref{sec:ctf_3d}) to reconstruct the tomograms.
\end{enumerate}

% Average defocus search

\subsubsection{Average power spectrum} \label{sec:algo:defocus_estimate:avg_ps}

We estimate the defocus by fitting theoretical CTFs, with varying defoci, to the power spectrum of the image. The quality of the fitting vastly depends on the magnitude of the Thon rings relative to the background. In a tilted image, the defocus ramp decreases the magnitude of the Thon rings, ultimately reducing the number of Thon rings in the power spectrum, thus reducing the accuracy of the defocus estimate. To reduce the negative interference between regions with different defoci (i.e. to reduce the effect of the defocus ramp), we exclude from the calculation of the power spectrum the regions that are ``too high'' or ``too low'' (defined by \code{deltaZtolerance}) from the tilt axis. So, if the tilt axis is parallel to the $y$ axis, the selected coordinates should satisfy:
\begin{equation}
    -\code{deltaZtolerance} < x \tan(\bm{\alpha}) < +\code{deltaZtolerance}
\end{equation} % in practice its *-1, but here we don't care I think.
where $\bm{\alpha}$ is the tilt angle and the $x$ coordinates are centered, meaning that the center of the axis (i.e. where the tilt axis is) is equal to $0$.
    
For each image, the tiles within the selected regions are extracted, padded by a factor of 2 and the average 2D power spectrum of these tiles $|\bm{W}_{exp}(q_{hk},\phi)|$ is calculated. By summing all of the average 2D power spectra of the tilt-series, we now have one 2D average power spectrum gathering the signal of only the regions ``close'' to the tilt axis in $z$.

\subsubsection{Average defocus} \label{sec:algo:defocus_estimate:avg_defocus}

This first search consists into finding a first rough estimate of the average defocus of the tilt-series, at the tilt axis. For now, we will ignore the astigmatism and compute the radial average of $|\bm{W}_{exp}(q_{hk},\phi)|$. We refer to this radial average as $|\bm{W}_{exp}(q_{h})|$. In order to find the defocus value $\bm{\mathrm{z}}$, we are going to fit a theoretical CTF curve against this radial average. To do so, a range of defocus value is going to be tested, such as for each $i^{th}$ defocus $\bm{\mathrm{z}}_{i} = \code{defEstimate} \pm \code{defWindow}$, 1nm increment:
\begin{enumerate}
    \item \textbf{Get the CTF}: Calculate the 1D theoretical CTF, $\bm{W}_{i}(q_{h})$, in 1/\si{\angstrom}. Note that the defocus $\bm{\mathrm{z}}_i$ is the only variable from one iteration to another.
    \begin{gather}
        \lambda = {1226.39\times10^{-2}}/{\sqrt{\code{VOLTAGE} + 0.97845\times10^{-6}\times\code{VOLTAGE}^2}}\notag\\
        E(q_h) = \frac{ e^{ -20{\left( q_h \times \code{PIXEL\_SIZE}\times10^{10} \right)}^{1.25}} + 0.1 } {1.1}\notag\\
        \bm{W}_{i}(q_h) = \sin{ \left( \frac{\pi}{2}\ \code{Cs}\ \lambda^{3}\ q_h^4 + \pi\ \lambda\ \bm{\mathrm{z}}_i\ q_h^2 - \code{AMPCONT} \right)}\ E(q_h)\label{eq:ctf}
    \end{gather} % maybe I should make the Planck constant etc. appear in the equation... % \frac{ 0.5 }{ h_{max}
    where $\lambda$ is the relativistic wavelength of the electrons, in \si{\angstrom}. $E$ is an ad-hoc envelope, in 1/\si{\angstrom}, down-weighting higher frequencies. \code{Cs} is the spherical aberration constant, in \si{\angstrom}.
        
    \item \textbf{Background estimate}: if $\bm{\mathrm{z}}_i$ is correct, the zeros of $|\bm{W}_i(q_h)|$ should be at the background level. Therefore, we extract the values of $|\bm{W}_{exp}(q_h)|$ at the $q_h$ frequencies where the zeros should be, i.e. where $\bm{W}_i(q_h)=0$. A smooth curve is fitted to these sampled points and used as the background estimate $\bm{B}_{exp}(q_h)$. We refer to the background-subtracted radial average as $|\bm{W}_{exp-B}(q_h)|$. The position of the zeros varies depending on the defocus, therefore the closer we get from the true defocus, the more accurate the background estimate, the greater the correlation between $|\bm{W}_i(q_h)|$ and $|\bm{W}_{exp-B}(q_h)|$.
        
    \item \textbf{Calculate the CC score}: Only the frequencies, referred as $q_{h'}$, from slightly before the first zero to the first zero after \code{defCutoff} are considered. The CTF curves are normalized and the normalized cross-correlation is calculated as follow:
    \begin{equation}
        \bm{\mathrm{CC}}_{i} = \frac{
            \sum_{q_{h'}=1}^{Q_{h'}}\ {|\bm{W}_i(q_{h'})|}\ |\bm{W}_{exp-B}(q_{h'})|
            }{ Q_{h'}\ \bm{\sigma}_{i}\ \bm{\sigma}_{exp-B} }
    \end{equation}
    where $Q_{h'}$ is the number of total $q_{h'}$ frequencies. $\bm{\sigma}_{i}$ and $\bm{\sigma}_{exp-B}$ are the standard deviations of $|\bm{W}_{i}|$ and $|\bm{W}_{exp-B}|$ within the $q_{h'}$ frequencies.
\end{enumerate}

The selected defocus, $\bm{\mathrm{z}}_{best}$, is the defocus that gave the best fit, i.e. the maximal $\bm{\mathrm{CC}}_{i}$. To help analysing the results, the following plots are saved:
\begin{itemize}
    \item \code{fixedStacks/ctf/<prefix>\_ali1\_ccFIT.pdf}: every $\bm{\mathrm{CC}}_i$ for every sampled $\bm{\mathrm{z}}_i$.
    \item \code{fixedStacks/ctf/<prefix>\_ali1\_bgFit.pdf}: the radial average (before background subtraction) $|\bm{W}_{exp}|$ and the estimated background $\bm{B}_{exp}$ computed using $\bm{\mathrm{z}}_{best}$, as a function of $q_{h'}$ frequencies. For visualization, the background curve is slightly shifted down.
    \item \code{fixedStacks/ctf/<prefix>\_ali1\_psRadial\_1.pdf}: the background-subtracted radial average $|\bm{W}_{exp-B}|$ (in black) and the theoretical CTF $|\bm{W}_{best}|$ (in gree), that gave the best fit, as a function of $q_{h'}$ frequencies.
\end{itemize} \label{subsubsec:avf_def}

\subsubsection{Average astigmatic defocus}

Until now, we have used the 1D radial average of the 2D power spectrum, $|\bm{W}_{exp}(q_h)|$,  to estimate the defocus. In order to account for astigmatism, we will now use the 2D power spectrum, $|\bm{W}_{exp}(q_{hk}, \phi)|$, computed in section \ref{sec:algo:defocus_estimate:avg_ps}.

% We could have a background based on the current astigmastim. This is more expensive but we have all the tools to do so.
One difference from the initial search is that we'll use the same background throughout the search. As there could be astigmatism in the images, this is less precise, but it is enough for now and it will be refined later anyway. We do as follow:
\begin{enumerate}
    \item \textbf{Background estimate}: Using $\bm{\mathrm{z}}_{best}$, calculated at the previous step, we compute $\bm{W}_{best}(q_h)$, as in equation \ref{eq:ctf}. Then, we extract from $|\bm{W}_{exp}(q_{hk})|$ 18 lines positioned at different angles (from 2.5\textdegree\ to 87.5\textdegree, 5\textdegree\ increment). For each of the 18 lines, we extract the CTF values of $|\bm{W}_{exp}|$ at the $q_{hk}$ frequencies where the zeros should be, i.e. where $|\bm{W}_{best}(q_{h})|=0$ and fit a smooth curve along these values. These curves are averaged and a 2D cubic spline is fitted to this average. This is the 2D background estimate, $\bm{B}_{exp}(q_{hk})$ and we refer to the background-subtracted power spectrum as $|\bm{W}_{exp-B}(q_{hk})|$.
        
    \item $|\bm{W}_{exp}|$ and $|\bm{W}_{exp-B}|$ are saved as \code{<prefix>\_avgPS.fixed} and \code{<prefix>\_avgPS-bgSub.fixed}, in \code{fixedStacks/ctf}. Only the $q_{h'k'}$ frequencies are shown, as they are the only ones to be considered during the search.
\end{enumerate}
    
Once the 2D-background-subtracted power spectrum is calculated, we can start the defocus search. The defocus value $\bm{\mathrm{z}}_{best}$ will stay unchanged during this search. The goal here is to find the defocus shift $\bm{\Delta\mathrm{z}}_{ast}$ and the astigmatic angle $\bm{\phi}_{ast}$. For this search, $\bm{\Delta\mathrm{z}}_{ast}$ is a range of $i$ values, from -200nm to 200nm, with 15nm steps and $\bm{\phi}_{ast}$ is a range of $j$ values, from -45\textdegree\ to 45\textdegree, with 10\textdegree\ step:

\begin{enumerate}
    \item \textbf{Get the CTF}: Calculate the 2D CTF. This is similar to equation \ref{eq:ctf}, but now the defocus varies with the direction of the azimuthal angle $\phi$, such as:
    \begin{gather}
        \bm{\mathrm{z}}_{i,j}(\phi) = (\bm{\mathrm{z}}_{best} - {[\bm{\Delta \mathrm{z}}_{ast}]}_i) \cos(\phi - {[\bm{\phi}_{ast}]}_j) + (\bm{\mathrm{z}}_{best} + {[\bm{\Delta \mathrm{z}}_{ast}]}_i) \sin(\phi - {[\bm{\phi}_{ast}]}_j)\notag\\
        \bm{W}_{i,j}(q_{hk},\phi) = \sin{ \left( \frac{\pi}{2}\ \mathrm{Cs}\ \lambda^3\ q_{hk}^4 + \pi\ \lambda\ \bm{\mathrm{z}}_{i,j}(\phi)\ q_{hk}^2 - \code{AMPCONT} \right)}\ E(q_{hk})\label{eq:ctf2}
    \end{gather}

    \item \textbf{Calculate the CC score}: Only the frequencies, referred as $q_{h'k'}$, from slightly before the first zero to the first zero after \code{defCutoff} are considered. The CTF curves are normalized and the normalized cross-correlation is calculated as follow:
    \begin{equation}
        \bm{\mathrm{CC}}_{i,j} = \frac{
                \sum_{q_{h'k'}=1}^{Q_{h'k'}}\ |\bm{W}_{i,j}(q_{h'k'},\phi)|\ |\bm{W}_{exp-B}(q_{h'k'},\phi)|
                }{ Q_{h'k'}\ \bm{\sigma}_{i,j}\ \bm{\sigma}_{exp-B} }
    \end{equation}
        where $Q_{h'k'}$ is the number of total $q_{h'k'}$ frequencies. $\bm{\sigma}_{i,j}$ and $\bm{\sigma}_{exp-B}$ are the standard deviations of $|\bm{w}_{i,j}|$ and $|\bm{w}_{exp-B}|$ within the $q_{h'k'}$ frequencies.
\end{enumerate}
    
The astigmatic defocus, referred as ($\bm{\mathrm{z}}_{best}$, $\bm{\Delta\mathrm{z}}_{best}$, $\bm{\phi}_{best}$), is the defocus that gave the best fit, i.e. the maximal $\bm{\mathrm{CC}}_{i,j}$ score.
    
Another defocus search is done on the power spectrum, with a finer sampling around the current best defocus. $\bm{\Delta\mathrm{z}}_{ast}$ is now a range of $i'$ values, from $\bm{\Delta\mathrm{z}}_{best}-$ 7.5nm to $\bm{\Delta\mathrm{z}}_{best}+$ 7.5nm, with 3.75nm steps and $\bm{\phi}_{ast}$ is now a range of $j'$ values, from $\bm{\phi}_{best}-$ 5\textdegree\ to $\bm{\phi}_{best}+$ 5\textdegree, with 0.5\textdegree\ step. The refined astigmatic defocus, referred as ($\bm{\mathrm{z}}_{best}$, $\bm{\Delta\mathrm{z}}_{best'}$, $\bm{\phi}_{best'}$), is the defocus that gave the best fit, i.e. the maximal $\bm{\mathrm{CC}}_{i',j'}$ score.

\subsubsection{Handedness check}

By convention, the defocus is a positive value. As such, if the sample is tilted, the regions higher than the tilt axis have a smaller defocus and the regions lower than the tilt axis have a larger defocus. If it is the opposite, it means the projections were flipped or the tilt-angles are not in the correct order or the tilt axis used for alignment is 180\textdegree\ off. In section \ref{sec:algo:defocus_estimate:avg_ps}, we have selected regions at about the same defocus than the tilt axis ($\pm \ \code{deltaZtolerance}$) to reduce interference in the radial average between regions with "significantly" different defoci. Here, we do the same, but we apply a $z$ shift to only select regions that are significantly above the tilt axis ($-\code{zShift}$), referred as $|\bm{W}_{-z}|$, or significantly below the tilt axis ($+\code{zShift}$), referred as $|\bm{W}_{+z}|$.

As described in section \ref{sec:algo:defocus_estimate:avg_defocus}, we have an estimate of the defocus value at the tilt-axis, referred as $\bm{\mathrm{z}}_{best}$, but we can also estimate the defocus for both $|\bm{W}_{-z}|$ and $|\bm{W}_{+z}|$. As such, we calculate the defocus estimate of the regions of the specimen that were imaged while being above the tilt axis (or at least we assume so), referred as $\bm{\mathrm{z}}_{-z}$, and a defocus estimate of the regions of the specimen that were imaged while being below the tilt axis (again, we assume so), referred as $\bm{\mathrm{z}}_{+z}$. As in section \ref{sec:algo:defocus_estimate:avg_defocus}, the fits are saved in \code{fixedStacks/ctf/<prefix>\_ali1\_psRadial\_2.pdf} and \code{fixedStacks/ctf/<prefix>\_ali1\_psRadial\_3.pdf}, respectively.

The regions below the tilt axis should be farther away from the focus compared to the regions above the tilt axis, thus if $\bm{\mathrm{z}}_{-z} < \bm{\mathrm{z}}_{best} < \bm{\mathrm{z}}_{+z}$, then the handedness is probably correct. On the other hand, if $\bm{\mathrm{z}}_{-z} > \bm{\mathrm{z}}_{best} > \bm{\mathrm{z}}_{+z}$, the handedness is probably wrong.

\subsubsection{Per-view astigmatic defocus} \label{sec:algo:defocus_estimate:per_view_defocus}

So far, every 2D power spectra and 1D radial averages were the result of an average across the entire tilt-series. Therefore, we could only estimate an average defocus and average astigmatism. In practice, each view can be at a different defocus and have a different astigmatism. Consequently, we should work on one projection at a time.

We have to deal mostly with tilted images, which inherently have less Thon rings than non-tilted images, making the CTF fitting more complicated. Moreover, there is considerably less signal in one single image than in the entire tilt-series. As such, excluding regions with a ``significantly'' different defocus, as in section \ref{sec:algo:defocus_estimate:avg_ps}, is not possible.
    
Fortunately, if an image is tilted with a known tilt, we can account for its defocus ramp and correct for it. By stretching/compressing the power spectrum of different sub-regions (i.e. tiles) of a tilted image, we can make them all constructively interfere with each other. As a result, the analysis of tilted specimens becomes almost as good as for non-tilted specimens, up to the point where the signal is getting weaker due to the increased thickness.
    
Padding an image in real space is effectively stretching its power spectrum. On the other hand, cropping an image in real space is effectively compressing its power spectrum. As such, with the correct padding/cropping factor, one can make two power spectra with different defoci constructively interfere. The tiles below the tilt axis have a larger defocus, so to make their power spectrum constructively interfere with the tiles at the tilt axis, it is likely that we'll need to pad them in order to stretch their power spectrum. On the other hand, the tiles above the tilt axis have a smaller defocus, so we'll need to crop the tiles in order to compress their power spectrum.

To prevent loosing some information during cropping, all the tiles are padded in advance to \code{paddedSize} (default=768), to make sure the cropping will not intrude actual data. As such, the goal of this section is to find, for each view $i$ and for each tile $j$, the padding factor $\bm{f}_{i,j}$, which, by stretching the power spectrum of the tile, maximizes the constructive interference between the power spectrum of the tile and the power spectrum at the tilt-axis. The CTF at the tilt-axis, $\bm{W}_{tilt}$, is defined by $\bm{\mathrm{z}}_{best}$, the average defocus at the tilt-axis calculated in section \ref{sec:algo:defocus_estimate:avg_defocus}. The output size of the tile $j$ from view $i$, is defined as:
\begin{equation} \label{eq:scale_size}
    \bm{X}_{i,j} = \mathrm{floor} (\code{paddedSize} \times \bm{f}_{i,j})
\end{equation}

\begin{note}If the \code{PIXEL\_SIZE} is lower than 2\si{\angstrom}, the Thon rings start to be compressed in a very small frequency window, which makes the analysis more complicated. In this case, we Fourier crop the tilt-series at 2\si{\angstrom}, effectively stretching the power spectrum of the image and making the CTF fitting easier.
\end{note}

% I am not talking about the first step where we exclude some highly tilted regions (regions that are exposed only in SOME view). TODO.

Each image is divided into overlapping tiles. For each view $i$ of the tilt-series and for each tile $j$:

\begin{enumerate}
    \item \textbf{Defocus shift}: We can calculate the defocus shift $\bm{\Delta \mathrm{z}}_{i,j}$ between a tile and the tilt-axis:
    \begin{equation}
        \bm{\Delta \mathrm{z}}_{i,j} = -x_{j}\tan(\bm{\alpha}_i)
    \end{equation}
    where $\bm{\alpha}_i$ is the tilt-angle, $x_{j}$ is the center of the tile in $x$. The $x$ coordinates are centered, meaning that the position of the tilt-axis in $x$ is 0. Of course, this assumes that the tilt-axis is parallel to the $y$ axis and that the defocus is a positive number. The defocus of the tile is defined as $\bm{\mathrm{z}}_{i,j} = \bm{\mathrm{z}}_{best} + \bm{\Delta \mathrm{z}}_{i,j}$ and we can calculate the CTF of the tile, $\bm{w}_{i,j}(q_h)$, as in equation \ref{eq:ctf}.

    \item \textbf{Padding factor estimate}: If we know $\bm{\Delta \mathrm{z}}_{i,j}$ and $\bm{\mathrm{z}}_{best}$, we can calculate a first estimate of the padding factor $\bm{f}_{i,j}$. One can note that if the tilt-angle is 0\textdegree, $\bm{f}_{i,j} = 1$ (i.e no padding).
    \begin{equation}
        \bm{f}_{i,j} = \sqrt{ \left( 1 + \frac{ \bm{\Delta \mathrm{z}}_{i,j} }{ \bm{ \mathrm{z}}_{best} } \right) }
    \end{equation}

    \item \textbf{Padding factor refinement}: At this point, $\bm{f}_{i,j}$ is just an estimate. The goal of this step is to refine $\bm{f}_{i,j}$ only based on how good the fit is between $\bm{W}_{i,j}(q_h)$ and $\bm{W}_{tilt}(q_h)$. As such, for each $\bm{f}_{i,j,k}$, within $\bm{f}_{i,j}-1$ up to $\bm{f}_{i,j}+1$, with a 0.001 $k$ step:
    \begin{enumerate}
        \item Stretch $\bm{W}_{i,j}(q_h)$ by a factor of $\bm{f}_{i,j,k}$, using linear interpolation.
        \item Calculate the normalized cross-correlation $\bm{\mathrm{CC}}_{i,j,k}$ between the stretched $\bm{W}_{i,j,k}(q_h)$ and $\bm{W0}_{tilt}(q_h)$.
    \end{enumerate}
    The padding factor that gave the best fit, ${[\bm{f}_{best}]}_{i,j}$, is selected and further refined within ${[\bm{f}_{best}]}_{i,j}-0.01$ up to ${[\bm{f}_{best}]}_{i,j}+0.01$, with a 0.0001 $k'$ step. The final padding factor is referred to as ${[\bm{f}_{best'}]}_{i,j}$
\end{enumerate}

Each tile of each view is then padded, in real space, using their ${[\bm{f}_{best'}]}_{i,j}$, as in equation \ref{eq:scale_size}. For each view, the average power spectrum of the padded tiles is calculated and stacked into \code{fixedStacks/ctf/<prefix>\_ali1-PS.mrc}. These power spectra are band-pass filtered (saved as \code{*ali1-PS2.mrc}), as discussed in section \ref{sec:algo:defocus_estimate:avg_defocus}, to only include the frequencies from slightly before the first zero to the first zero after \code{defCutoff}.

Finally, these band-passed power spectra are sent to \href{https://grigoriefflab.umassmed.edu/ctffind4}{CTFFIND 4} for analysis, resulting into a per-view defocus value $\bm{\mathrm{z}}_{i}$, a per-view astigmatic defocus shift ${[\bm{\Delta\mathrm{z}}_{ast}]}_{i}$ and a per-view astigmatic angle ${[\bm{\phi}_{ast}]}_{i}$. The outputs for each view $i$ is saved in the following table.
\renewcommand{\arraystretch}{1.2}
\begin{longtable}[c]{| l | p{35mm} || l | p{35mm} || l | p{35mm} |}
\captionsetup{labelfont=bf}
\caption{\code{fixedStacks/ctf/<prefix>\_ali*\_ctf.tlt}} \label{tab:ctf_tlt}\\
% I need to check the diff between first and last column. It is useful when removing images.

\hline
\textbf{C} & \textbf{Description} & \textbf{C} & \textbf{Description} & \textbf{C} & \textbf{Description}\\
\hline
1 & index $i$                   & 9 & $\bm{R}_{1,2,i}$               & 17 & \code{Cs}\\
\hline
2 & $\bm{T}_{x,i}$ (in pixels)  & 10 & $\bm{R}_{2,2,i}$              & 18 & \code{WAVELENGTH}\\
\hline
3 & $\bm{T}_{y,i}$ (in pixels)  & 11 & $i^{th}$ post-exposure       & 19 & \code{AMPCONT}\\
\hline
4 & $\bm{\alpha}_i$             & 12 & ${[\bm{\Delta\mathrm{z}}_{ast}]}_i$    & 20 & pixels in X\\
\hline
5 & \cellcolor{lightgray}empty  & 13 & ${[\bm{\phi}_{ast}]}_i$                & 21 & pixels in Y\\
\hline
6 & 90\textdegree               & 14 & \cellcolor{lightgray}empty   & 22 & sections Z\\
\hline
7 & $\bm{R}_{1,1,i}$             & 15 & $\bm{\mathrm{z}}_i$          & 23 & ?\\
\hline
8 & $\bm{R}_{2,1,i}$             & 16 & \code{PIXEL\_SIZE}           & 24 & \cellcolor{lightgray}\\
\hline
\end{longtable}

\newpage

\subsection{Template matching} \label{sec:algo:picking}

\subsubsection{Pre-processing the tomogram}

%% RESAMPLE THE TILT-SERIES
% 1) extract every tilt and sub-region from the metadata.
% 2) If the binned stacks do not exist, bin the aligned stacks, in series.
%       - get the header of the aliStack
%       - calculate the new pixel size and dimensions
%       - multiply each projection with 1/sinc(gX)^2: amplify the corners
%       - bandpass: highcut: 600, lowcut: binned pixel size
%       - resample image in Fourier space (rotation + eventual shifts to keep the center the same) and cut the final image (in real space). Save them in cache/
%       - do this for every stack.

The desired sub-region tomogram $\bm{V}$ is reconstructed by weighted back-projection using {\tilt}.
 \begin{enumerate}
    \item \textbf{Get the aligned stack}: The aligned stack saved in \code{aliStacks} is loaded and binned to the desired sampling (i.e. \code{Tmp\_sampling}) as described in section \ref{sec:algo:ctf_3d:resample}. If the stack already exists in \code{cache}, it is not recalculated.
    
    \item \textbf{Reconstruct the subregion tomogram}: The sub-region coordinates are extracted from table \ref{tab:recon_coords} and used to set the {\tilt} entries \code{SLICE}, \code{THICKNESS} and \code{SHIFT}. The tilt angles saved in table \ref{tab:ctf_tlt} are used for the \code{-TILTFILE} entry. If there are local alignments, the \code{.local} file is used for the \code{-LOCALFILE} entry. The output reconstructions from {\tilt} are oriented with the $y$ axis in the third dimension. With \href{https://bio3d.colorado.edu/imod/doc/man/trimvol.html}{trimvol} \code{-rx} entry, we rotate by -90\textdegree\ around $x$ to place the $z$ axis in the third dimension.
    \begin{note}This step is a simpler version of the reconstruction described in section \myref{sec:algo:ctf_3d}.\end{note}
\end{enumerate}

If the sub-region is larger than \code{Tmp\_targetSize}, it is divided into $c$ equal chunks. For each chunk $c$:
\begin{enumerate}
    \item \textbf{Band-pass filter}: The chunk $\bm{V}_c$ is band-pass filtered by $\bm{W}_{bandpass}$, which has a high-pass cutoff at 600\r{A} to remove ice/intensity gradients and a low-pass cutoff a \code{lowResCut} or if it is not defined a cutoff at the first CTF zero. The zero is estimated based on the average defocus value of the stack (table \ref{tab:ctf_tlt}). The chunk is then centered and standardized.
    \begin{note}If \code{Tmp\_medianFilter} is defined, the chunk $\bm{V}_c$ is median filtered using the specified neighborhood window.\end{note}
    
    \item \textbf{Positive contrast}: The chunk is low-pass filtered and we assume that \code{lowResCut} cuts before the first zero of the CTF. As such, the negative contrast is ``flipped'' in real-space by simply multiplying the chunk by $-1$.
\end{enumerate}
Finally, the variance of $\bm{V}$ is calculated and used to normalize each chunk $\bm{V}_c$.

\subsubsection{Pre-processing the template}
The template $\bm{S}$ is loaded, padded up to $2\times\code{Ali\_mRadius}$ while enforcing a squared box. It is then centered, standardized and finally resampled to \code{Tmp\_sampling}, using linear interpolation.

\subsubsection{Angular search}
% Computing the cross-correlation (CC) between the tomogram and the template gives us a CC map. Each pixel of this map corresponds to the CC score between the template and the tomogram centered on this pixel. If we sample different rotations, how do we keep track of CC scores? After all, the CC scores do not tell us what was the rotation of the template.

The in- and out-of-plane angles $[\Theta_{out},\ \Delta_{out}, \Theta_{in},\ \Delta_{in}]$ registered in \code{Tmp\_angleSearch} are converted into a set of $r \times 3$ Euler angles ($\phi_{r},\ \theta_{r},\ \psi_{r}$). These rotations are finally converted into $r$ rotation matrices $\bm{R_{r}}$.

To keep track of things, two empty volumes of the same size as the chunks are prepared, for each chunk. $\bm{\mathrm{CC}}_{best\text{-}peak}$ will store the standardized cross-correlation scores and $\bm{\mathrm{CC}}_{best\text{-}rot}$ will store the index $r$.

For each chunk $c$ and for each rotation $r$:
\begin{enumerate}
    \item \textbf{Rotate and pad the template}: The template is rotated by $\bm{R}_{r}$, padded to the size of $\bm{V}_c$, band-pass filtered with $\bm{W}_{bandpass}$ and any change in power due to interpolation is corrected. We refer to this transformed template as $\bm{S}_r$.
    
    \item \textbf{Calculate the normalized cross-correlation}: The cross-correlation between the tomogram chunk $\bm{V}_c$ and the rotated template $\bm{S}_r$ is calculated as follow.
    \begin{equation}
        \bm{\mathrm{CC}}_{c,r} =    \mathcal{F}^{-1} \left\{
                                        \mathcal{F} \left\{ \bm{V}_c \right\} \overline{\mathcal{F} \left\{ \bm{S}_r \right\}}
                                    \right\}
    \end{equation}
    Importantly, $\bm{\mathrm{CC}}_{c,r}$ is then normalized by its standard deviation. % add more detail on this.
    
    \item \textbf{Update the best score}: We only want to save the best peaks, i.e. the peaks that are higher than the previous iterations. As such, each voxel $v$ of ${[\bm{\mathrm{CC}}_{best\text{-}peak}]}_c$ and ${[\bm{\mathrm{CC}}_{best\text{-}rot}]}_c$ are updated as follow:
    \begin{equation}
        {[\bm{\mathrm{CC}}_{best\text{-}peak}(v)]}_c =
            \begin{cases}
                \bm{\mathrm{CC}}_{c,r}(v), & \text{if}\ {[\bm{\mathrm{CC}}_{best\text{-}peak}]}_c(v) \leqslant  \bm{\mathrm{CC}}_{c,r}(v)\\
                {[\bm{\mathrm{CC}}_{best\text{-}peak}]}_c(v), & \text{otherwise}
            \end{cases}
    \end{equation}
    \begin{equation}
        {[\bm{\mathrm{CC}}_{best\text{-}rot}]}_c(v) =
            \begin{cases}
                r, & \text{if}\ {[\bm{\mathrm{CC}}_{best\text{-}peak}]}_c(v) \leqslant \bm{\mathrm{CC}}_{c,r}(v)\\
                {[\bm{\mathrm{CC}}_{best\text{-}rot}]}_c(v), & \text{otherwise}
            \end{cases}
    \end{equation}
    Consequently, each voxel $v$ is assigned to the current best CC score and to the rotation $r$ that gave this score.
    \begin{note}Of course, if it is the first iteration, i.e. if $r=1$, then ${[\bm{\mathrm{CC}}_{best\text{-}peak}]}_c = \bm{\mathrm{CC}}_{c,r}$ and ${[\bm{\mathrm{CC}}_{best\text{-}rot}]}_c = 1$.
    \end{note}
\end{enumerate}
    
At the end of the angular search, the chunks are concatenated and the following files are saved in \code{convmap\_wedgeType\_2\_bin<X>} (\code{<X>} is equal to \code{Tmp\_sampling}):
\begin{itemize}
    \item \code{<prefix>\_<region>\_bin<X>\_convmap.mrc}: The best scores, i.e. $\bm{\mathrm{CC}}_{best\text{-}peak}$.
    \item \code{<prefix>\_<region>\_bin<X>\_angles.mrc}: The corresponding rotation $r$ for each best score, i.e. $\bm{\mathrm{CC}}_{best\text{-}rot}$.
    \item \code{<prefix>\_<region>\_bin<X>\_angles.list}: The $(\phi,\ \theta,\ \psi)$ Euler angles corresponding to the rotation $r$. One trio per line, $r$ lines.
\end{itemize}

\subsubsection{Extract the peaks} \label{sec:algo:picking:extract_peaks}

The goal now is to select the $x,\ y,\ z$ coordinates of the $p$ strongest peaks registered $\bm{\mathrm{CC}}_{best\text{-}peak}$, with $p$ equal to \code{Tmp\_threshold}. Then, for each peak, the corresponding rotation $r$ is extracted from $\bm{\mathrm{CC}}_{best\text{-}rot}$ and converted back to the corresponding $(\phi,\ \theta,\ \psi)$ Euler angles. Therefore, for each desired peak $p$:
\begin{enumerate}
    \item \textbf{Get the coordinates}: The $x,\ y,\ z$ coordinates ${[\bm{T}_{x,y,z}]}_p$ of the strongest peak registered in $\bm{\mathrm{CC}}_{best\text{-}peak}$ are selected. To take into account the neighbouring pixels, these coordinates are adjusted by the local center of mass ($3\times3\times3$ matrix, centered on the strongest peak). ${[\bm{T}_{x,y,z}]}_p$ is relative to the subregion tomogram $\bm{V}$, with the origin at the lower left corner and is unbinned.
    % mouais...
    
    \item \textbf{Get the rotation}: The value of $\bm{\mathrm{CC}}_{best\text{-}rot}$ at the coordinates ${[\bm{T}_{x,y,z}]}_p$ is extracted. This value is the rotation index $r$ and is converted back to the corresponding $(\phi_p,\ \theta_p,\ \psi_p)$ Euler angles and rotation matrix $\bm{R}_p$. If a symmetry was entered, the rotation is randomized between the symmetry related pairs to reduce missing-wedge bias.
    
    \item \textbf{Get the CC score}: The value of the peak in $\bm{\mathrm{CC}}_{best\text{-}peak}$ at position ${[\bm{T}_{x,y,z}]}_p$ is extracted, centered and standardized. This score is referred as $\bm{\mathrm{CC}}_p$.
        
    \item \textbf{Erase the selected peak}: The selected peak at position ${[\bm{T}_{x,y,z}]}_p$ and its neighbouring voxels are masked-out by $\bm{M}_{peak}$. $\bm{M}_{peak}$ is set by \code{Peak\_mType} and \code{particleRadius} (or \code{Peak\_mRadius} if it is defined), and is rotated by $\bm{R}_p$ before being applied. In that way, the next iteration cannot select the same peak nor the peaks within this particle radius.

    \item \textbf{Save the peak information in table \ref{tab:csv}}.
\end{enumerate}
Finally, the coordinates $\bm{T}_{x,y,z}$ are binned to \code{Tmp\_sampling} and saved into \code{<prefix>\_<region>\_bin<X>.pos} and converted into an IMOD mod file with the following command:
\begin{lstlisting}
point2model -number 1 -sphere 3 -scat *.pos  *.mod
\end{lstlisting}

\renewcommand{\arraystretch}{1.2}
\begin{longtable}[c]{| l | p{29mm} || l | p{29mm} || l | p{29mm} || l | p{29mm} |}
\captionsetup{labelfont=bf}
\caption[\code{convmap/<prefix>\_<region>\_bin<X>.csv}]{\code{convmap\_wedgeType\_2\_bin<X>/<prefix>\_<region>\_bin<X>.csv}. One line per particle $p$. The translations ($\bm{T}_x$, $\bm{T}_y$, $\bm{T}_z$) are in pixel, un-binned. The Euler angles ($\phi$, $\theta$, $\psi$) are described in section \ref{sec:algo:euler_conventions}. They are actually not directly used by {\emClarity}. As mentioned previously, the rotation matrices ($\bm{R}_{m,n}$, $m=$ rows, $n=$ columns) are meant to be applied to the particles to rotate them from the microscope frame to the reference frame. In this case, the translations are applied before the rotation.} \label{tab:csv}\\

\hline
\textbf{C} & \textbf{Description} & \textbf{C} & \textbf{Description} & \textbf{C} & \textbf{Description} & \textbf{C} & \textbf{Description}\\
\hline
1 & $\bm{\mathrm{CC}}_p$                & 8 & \cellcolor{lightgray} empty (1)     & 15 & $\theta_p$                & 22 & ${[\bm{R}_{32}]}_p$\\
\hline
2 & \code{Tmp\_sampling}                & 9 & \cellcolor{lightgray} empty (1)     & 16 & $\psi_p$                 & 23 & ${[\bm{R}_{13}]}_p$\\
\hline
3 & \cellcolor{lightgray} empty (0)     & 10 & \cellcolor{lightgray} empty (0)    & 17 & ${[\bm{R}_{11}]}_p$      & 24 & ${[\bm{R}_{23}]}_p$\\
\hline
4 & Unique ID, $p$                      & 11 & ${[\bm{T}_x]}_p$                   & 18 & ${[\bm{R}_{21}]}_p$      & 25 & ${[\bm{R}_{33}]}_p$\\
\hline
5 & \cellcolor{lightgray} empty (1)     & 12 & ${[\bm{T}_y]}_p$                   & 19 & ${[\bm{R}_{31}]}_p$      & 26 & \cellcolor{lightgray} Class (1)\\
\hline
6 & \cellcolor{lightgray} empty (1)     & 13 & ${[\bm{T}_z]}_p$                   & 20 & ${[\bm{R}_{12}]}_p$      & \cellcolor{lightgray} & \cellcolor{lightgray}\\
\hline
7 & \cellcolor{lightgray} empty (1)     & 14 & $\phi_p$                           & 21 & ${[\bm{R}_{22}]}_p$      & \cellcolor{lightgray} & \cellcolor{lightgray}\\
\hline

\end{longtable}

\newpage

\subsection{Tomogram reconstruction} \label{sec:algo:ctf_3d}

%%%%% Workflow

%% Parameters:
% applyExposureFilter (default=1)
% useSurfaceFit (default=1)
% flgDampenAliasedFrequencies (default=0)
% flg2dCTF (default=0): force to 1 section equal to maxZ

% Ali_samplingRate
% PIXEL_SIZE

\subsubsection{Resample the tilt-series} \label{sec:algo:ctf_3d:resample}

For efficiency and practicality, we often start working with ``binned'' data and progressively decrease the ``binning'', up to the point where the original data is used. Binning consists into reducing images by an integer multiple to ensure that every output pixel is an average of the same number of neighbouring pixels (the ``bin''). There are many ways to bin an image: \href{https://www.imagemagick.org/Usage/filter/#box}{box filter}, \href{https://entropymine.com/imageworsener/pixelmixing/}{pixel mixing}, Fourier cropping, interpolation, etc. Currently, {\emClarity} bins the tilt-series by resampling them in a new, smaller, coordinate grid (in Fourier space), using bilinear interpolation.

Therefore, the original aligned tilt-series, saved in \code{aliStacks} and registered in the metadata during the project initialization (section \ref{sec:init}), are resampled, in parallel, as follow:
\begin{enumerate}
    \item TODO: $1/{\mathrm{sinc}(gX)}^2$.
    % This should just be a note.
    % This is a real problem. Fourier crop should just work.
    % resample2d is updated, so update this.

    \item The projections are Fourier transformed and band-pass filtered. Low frequencies are removed up to $\sim$600\r{A}, which sets the mean of the projections to 0 and attenuates eventual large intensity gradients present in the images. Frequencies after the new binned pixel size $\bm{p}_{bin}$ are also removed.  $\bm{p}_{bin}$ is equal to \code{PIXEL\_SIZE} $\times$ \code{Ali\_mSamplingRate}.

    \item When resizing images, we force the output image to fit into a new grid. This operation ultimately defines a new center in pixel space and it is important to keep this new center aligned with the center of the original image, so that any operation done on the binned data can be scaled back to the original data. Depending on the binning factor and image size, we can anticipate and shift the images before scaling, to keep the new center aligned with the original grid.

    \item The projection spectra are scaled by $\bm{p}_{bin}$ using bilinear interpolation and the inverse Fourier transform is calculated to switch the images back to real space. Remember that scaling the frequency spectrum of an image by $n$ is equivalent to scaling the image by $1/n$.

    \item At this point, the images are ``zoomed out``, but the actual size of the images are unchanged. To complete the process of resizing the image, we need to crop them to the desired binned size.
\end{enumerate}

The binned stacks are saved in the \code{cache} directory as \code{<prefix>\_<ali1>\_bin<nb>.fixed}.

% 3) Check that the reconstruction at this binning doesn't exist already. If every subregions from one stack are already reconstructed, then skip it.

\subsubsection{Defocus step} \label{sec:algo:ctf_3d:defocus_step}

Defocus-gradient corrected back-projection, as described in \cite{jensen_3dctf}, requires that ``during the reconstruction of tomogram by [weighted] back-projection, each voxel is calculated from tilted images that were CTF-corrected with defocus values corresponding to the position of that voxel at each tilt. To achieve this, each image in the tilt-series is CTF-corrected multiple times with different defocus value. The number of different CTF corrections performed per image depends upon how finely the defocus gradient should be sampled'' \cite{novaCTF}.

\begin{note}During tilt-series alignment, the tilt axis is aligned to the $y$ axis. As such, if we assume that the specimen is flat, the defocus only varies along the $x$ and $z$ axis.
\end{note}

In practice, the tomograms are divided into $s$ $z$-sections of equal width, also referred as $z$-slabs or simply sections. Each section is assigned to a CTF-corrected tilt-series with a defocus corresponding to the defocus at the center of the section. The sampling of the defocus gradient, set by the defocus step $\bm{\Delta \mathrm{z}}$, is defined to keep the average CTF amplitudes, resulting from the destructive interference between all of the CTFs within a section, above the current resolution target. It is calculated as follow:
\begin{enumerate}
    \item The specimen thickness $\bm{t}$ is defined by the $z_{min}$ and $z_{max}$ boundaries of the sub-regions of the current tilt-series. The sub-regions are defined manually, as described in section \ref{sec:subregions}.
    
    \item The goal of this procedure is to progressively decrease $\bm{\Delta \mathrm{z}}$, up to the point where the $z$ sampling is fine enough to achieve the resolution target. $\bm{\Delta \mathrm{z}}$ is initially set to the specimen thickness $\bm{t}$. First, we calculate a theoretical CTF for each $z$ point defined from $-{\bm{\Delta \mathrm{z}}}/2$ to ${\bm{\Delta \mathrm{z}}}/2$, with $0.1 \times \bm{\Delta \mathrm{z}}$ increment, and average all of these CTFs together. This gives us an estimate of the average CTF resulting from the interference of many CTFs along a $z$ section. The defocus estimate used to calculate the CTFs is the average of the defoci calculated in section \ref{sec:algo:defocus_estimate:per_view_defocus} and stored in table \ref{tab:ctf_tlt}.

    \item Using this average CTF, we can estimate the maximum resolution $\bm{h}_{max}$, in 1/\r{A}, that a subtomogram could achieve if we were to use the current $\bm{\Delta \mathrm{z}}$. $\bm{h}_{max}$ corresponds to the highest frequency, up to Nyquist, where the average CTF is above 90\% of contrast.
    %The reason why we take the highest frequency and not the first frequency where the CTF first goes below 90\% of contrast is bc a very wide range of defocus will be represented in the subtomograms due to the tilt, so the sampling function will be smoother without reaching 0... NOT SURE ABOUT THIS.

    \item We define the resolution target as $\bm{h}_{cut}/2$, where $\bm{h}_{cut}$ is the frequency cutoff, in 1/\r{A}, defined in equation \ref{eq:hcut}. If it is the first cycle and the half-maps are not reconstructed yet, $\bm{h}_{cut}$ is set to 40\r{A}. If $\bm{h}_{max} \geqslant \bm{h}_{cut}/2$, it indicates that the defocus gradient is probably sampled enough to achieve the resolution target. On the other hand, if $\bm{h}_{max} < \bm{h}_{cut}/2$, it indicates that we are likely to benefit from a finer sampling, i.e. a smaller $\bm{\Delta \mathrm{z}}$. In this case, we decrease the current $\bm{\Delta \mathrm{z}}$ by 90\% and recalculate the average CTF and $\bm{h}_{max}$. This procedure is repeated until $\bm{h}_{max} \geqslant \bm{h}_{cut}$, up to the minimum allowed value $\bm{\Delta \mathrm{z}} = 10$ nm.
    \begin{note}Since we expect the resolution to improve using this new reconstruction, the resolution of the new reconstruction must be higher than what we currently have. The value $\bm{h}_{cut}/2$ roughly balances the trade off between achievable resolution and run time during the reconstruction.
    \end{note}
    
    \begin{note}If $\bm{h}_{max} \geqslant \bm{h}_{cut}$ at the first iteration, when $\bm{\Delta \mathrm{z}}$ is equal to the specimen thickness $\bm{t}$, the CTF correction shouldn't be considered really as ``3D'' as we will only use one $z$ section. On the other hand, it also means that, given the resolution target, the specimen is thin enough to not likely benefit from a ``3D'' correction.
    \end{note}
\end{enumerate}

Once the defocus step $\bm{\Delta \mathrm{z}}$ and thickness of the specimen $\bm{t}$ are calculated, we can define the number $s$ of $z$-sections, as the closest odd integer from $\bm{t}\ /\ \bm{\Delta \mathrm{z}}$.

% \subsubsection{Z sections}

% Once $\Delta f$ and $T$ are calculated, we can define the number of z sections needed as the closest odd integer from $T\ /\ \Delta f$. A z section, simply referred as section, is nothing more than a z slab of the tomogram with a width equal to $\Delta f$. Each section will be reconstructed independently from the other sections.

% Therefore, for each section, we are going to calculate one tilt-series that will be CTF-corrected using the section's defocus offset. Then, for each sub-region, we reconstruct each section independently using their respective tilt-series. Once the sections are calculated, we concatenate them to form the final 3D-CTF corrected sub-region tomograms. The ability to reconstruct only subsets of a bigger field-of-view heavily relies on the \code{SLICE}, \code{THICKNESS} and \code{SHIFT} parameters of the {\tilt} program.

% Before calculating the CTF-corrected tilt-series for each section of the specimen, we can calculate the spatial model of each section, which will make the CTF-correction more accurate.

\subsubsection{Center-of-mass and spatial model} \label{sec:algo:ctf_3d:spatial_model}

The CTF estimate, fitted from the power spectra, is considered to be the sum of CTFs from weak phase objects at varying defoci within the field of view. The center-of-mass of the specimen in $z$, $\bm{\mathrm{COM}}_z$, where most of the signal comes from, greatly impacts this average CTF, to the point where we can assume that the defocus estimate is the distance from $\bm{\mathrm{COM}}_z$ to the focal plane. The $z$-sections are positioned relative to the center of the reconstruction, $\bm{\mathrm{COR}}_z$, therefore to calculate their respective defocus and correctly estimate their average CTF, we must adjust the current defocus estimate to match the center of the reconstruction $\bm{\mathrm{COR}}_z$. In conclusion, we need to know the defocus at the center of the specimen or in other words, the average $z$-offset $\bm{\bar{Z}}_{R\text{-}M} = \bm{\mathrm{COR}}_z - \bm{\mathrm{COM}}_z$.

Here, we make the assumption that most of the signal comes from the particles, therefore that the defocus estimate is the distance from the center-of-mass of all the subtomograms to the focal plane. As the particle positions are expressed relative to $\bm{\mathrm{COR}}_z$ (i.e. $z=0$ at the center of the reconstruction), it becomes very easy to calculate $\bm{\bar{Z}}_{R\text{-}M}$. Moreover, as we know which particle belongs to which section, we can calculate an offset for each section $s$. We refer to these offsets as ${[\bm{\bar{Z}}_{R\text{-}M}]}_{s}$.

Of course, the particles within a section are not necessarily into the same $z$ plane, meaning that ${[\bm{Z}_{R\text{-}M}]}_s$ varies with the $x$ and $y$ coordinates. To take this into account, we calculate a spatial model describing the $z$-positions of the particles across the specimen. To do so, we extract the $x,\ y,\ z$ coordinates of the particles of every sub-regions of the specimen and, according to their $z$-position, the particles are assigned to a section. If a section contains more than 6 particles, we fit a quadratic surface to the particle positions. This surface defines the ``spatial model'' of the section or in other words ${[\bm{Z}_{R\text{-}M}]}_s(x,y)$. If there is less than 6 particles, the spatial model is an horizontal plane (it is the average $z$-position of the particles and is invariant across the section, i.e. ${[\bm{\bar{Z}}_{R\text{-}M}]}_s$).

% NOTE: Take the coordinates of every valid particle, bin the coordinates, recenter the coordinates (tilt-axis=0; shift from lower left to centered and include the tomo Z offset from the microscope frame) and add the Z coordinates of the particles. Do this for every sub-region in the current stack.

\subsubsection{3D-CTF phase correction} \label{sec:algo:ctf_3d:ctf_phase_correction}

%%%% Calculate the CTF phase corrected (multiply by CTF and exposure filter) tilt-series.
For each section $s$, we are going to calculate one CTF-corrected tilt-series using the section's defocus offset. Therefore, for each section $s$ and for each view of the tilt-series $i$:
% As we have said previously, we need to calculate a CTF-corrected tilt-series for each section:

\begin{enumerate}
    \item \textbf{Exposure filter}: Calculate the Fourier transform of the view and multiply it with the exposure filter ${[\bm{W}_{exposure}]}_i$, as in \cite{exposure_grant_2015}. This filter only varies with the tilt angle of the image and therefore is identical for each section.
    \begin{note}To take into account that the tilt-series, and therefore the subtomograms, are exposure filtered, the same filter will be applied to the sampling functions of the particles in section \myref{sec:algo:avg:SF3D}.
    \end{note}
    
    \item \textbf{Microscope frame}: The view is replaced within the microscope frame. To do so, the spatial model ${[\bm{Z}_{R\text{-}M}]}_s(x,y)$ is tilted according to the tilt angle of the current image. As described previously, because it is calculated from the $z$-positions of the particles, the spatial model of the section is already correctly positioned in $z$.
    
    \item \textbf{Defocus ramp}: The microscope frame is then divided into $n$ $z$-slabs of $\bm{\Delta \mathrm{z}}$ width, which effectively divides the view (represented by the transformed spatial model ${[\bm{Z}_{R\text{-}M}]}_{s,i}$) into $n$ strips parallel to the tilt-axis. This is similar to figure 1.A.right, from \cite{novaCTF} and forms a defocus ramp perpendicular to the tilt-axis. The strips follow a defocus ramp, defined as follow:
    \begin{equation} \label{eq:def_ramp}
        \bm{\mathrm{z}}_{i} + {z'}_{min} - \bm{\Delta \mathrm{z}}
                                  \xrightarrow[up\ to]{+\bm{\Delta \mathrm{z}}}\
                                  \bm{\mathrm{z}}_{i} + {z'}_{max} + \bm{\Delta \mathrm{z}}
    \end{equation}
    where ${z'}_{min}$ and ${z'}_{max}$ are the highest and lowest $z$ coordinates of the transformed (i.e tilted) spatial model. $\bm{\mathrm{z}}_{i}$ is the defocus value of the current view, saved in table \ref{tab:ctf_tlt}.
    
    \begin{note}As the spatial model is not necessarily a plane, the strips can be ``curved'' (i.e. with a variable width along the tilt-axis).
    \end{note}
    
    \begin{note}For a 0\textdegree\ image, $\bm{\mathrm{z}}'_{max} + \bm{\mathrm{z}}'_{min} \leqslant \bm{\Delta \mathrm{z}}$, because the spatial model only takes into consideration the particles from within a $z$-section. In other words, the 0\textdegree \ spatial model ``fits'' into the central slab of $\bm{\Delta \mathrm{z}}$ width and therefore has only one strip. On the other hand, as the tilt-angle increases, more strip are necessary to fully cover the spatial model.
    \end{note}
    
    \item \textbf{CTF multiplication}: At this point, the spatial model is correctly positioned in the microscope frame and divided into $n$ strips. An array is allocated in memory to hold the final CTF-corrected view and it is progressively filled, one strip at a time. For each strip $n$:
    \begin{enumerate}
        \item We calculate the 2D CTF of the current strip. The astigmatic defocus of the strip depends on the spatial model and therefore contains the $z$-offset of the strip, the $z$-offset of the section and the $z$-offset to take into account that $\bm{\Delta \mathrm{z}}$ is the defocus at the center-of-mass of the specimen and not at the center of the reconstruction.

        \item A copy of the Fourier transform of the current image is multiplied by this 2D CTF and inverse Fourier transformed. The pixels that belongs to the current strip are extracted from this CTF-multiplied image and added to the pre-allocated output array.
        
        \begin{note}Adjacent strips can slightly overlap by 1 pixels. The values of these pixels are divided by 2 to eliminate this overlapping artefact. 
        \end{note}
    \end{enumerate}
    
    \item \textbf{Save the CTF-corrected stacks}: Once every images of the tilt-series are reconstructed, the stack of CTF-corrected images is temporarily saved in the \code{cache} directory.
\end{enumerate}

At this point of the procedure, we have calculated one CTF-corrected tilt-series for each $z$-section of the specimen.

% FOR EACH SECTION:
% 1) assume the defocus determined is the distance from the center of mass of subtomograms in Z to the focal plane, rather than the center of mass of the tomograms (specimen): add avgZ to the defocus (defocus offset). This is only done if the section hasn't a surfaceFit. If it does, the surfaceFit already has the offset.

% FOR EACH PROJECTION
% 1) pad projection to optimum FFT size.
% 2) Compute and apply the exposure filter.
% 3) Compute the mesh grids for X and Y. For the Z, use the surfaceFIT, otherwise zeros (flat).
% 4) Transform the specimen plane to take into account the tilt angle. The tZ is multiplied by pixel_size (meter opposed to pixel) and is shifted by the defocus value (which contains the defocus offset).
% 5) Extract the defocus ramp (min tZ, max tZ).
% 6) Using the same defocus step, define the strips: minDefocus-ctf3dDepth/1:ctf3dDepth/1:maxDefocus+ctf3dDepth/1.
%    For each strip:
%       - allocate correctedPrj zeros().
%       - compute the vector defocus of the strip (defocus_strip) and compute the CTF.
%       - multiply the entire projection with the CTF, save as tmpCorrection.
%       - define the strip mask: (tZ > defocus_strip - ctf3dDepth/2 & tZ <= defocus_strip + ctf3dDepth/2) which defines the region of the transformed image that belongs to the strip. The strips are not linear if the sample is not planar.
%       - Add tmpCorrection(mask) to correctedPrj.
%   We also make sure that if the strip overlap (rounding error, surface fit...), we'll take the mean value for these pixels sampled multiple times (usually max 2, it is at the edges of the strips).

% Save the corrected stack (multiplied by CTF and exposure filtered).

\subsubsection{Tomogram reconstructions} \label{sec:algo:ctf_3d:reconstruction}

For each sub-region, we reconstruct each $z$-section of the sub-region tomogram independently, using their respective CTF-corrected tilt-series. Once the $z$-sections are reconstructed, we concatenate them to form the final 3D-CTF corrected sub-region tomograms.
For each $z$-section $s$ of a given specimen:
\begin{enumerate}
    \item For each sub-region defined for this specimen, we reconstruct with {\tilt} the current section using the section's 3D-CTF corrected tilt-series. The tilt angles saved in table \ref{tab:ctf_tlt} are used for the \code{-TILTFILE} entry. If there is a \code{.local} file for this specimen, it will be assigned to the \code{-LOCALFILE} entry. The cosine stretching is turned-off (\code{-COSINTERP 0}), with no low-pass filtering \code{-RADIAL 0.5,0.05}.
    \begin{note}The ability to reconstruct only subsets of a bigger volume heavily relies on the \code{SLICE}, \code{THICKNESS} and \code{SHIFT} entries of the {\tilt} program.
    \end{note}

    \item The output reconstructions from {\tilt} are oriented with the $y$ axis in the third dimension. With \href{https://bio3d.colorado.edu/imod/doc/man/trimvol.html}{trimvol} \code{-rx} entry, we rotate by -90\textdegree\ around $x$ to place the $z$ axis in the third dimension.
\end{enumerate}

Once all the sections of each sub-region are reconstructed, the $z$-sections of the same sub-region are stacked together in $z$ with \href{https://bio3d.colorado.edu/imod/doc/man/newstack.html}{newstack} to create the final 3D-CTF corrected sub-region tomograms.

\newpage

\subsection{Subtomogram averaging} \label{sec:algo:avg}

% NOTE for Ben: This part is finished and can be reviewed.

For each half-set, this procedure computes the CTF-corrected filtered subtomogram average $\bm{S}$, via a volume normalized single-particle Wiener (SPW) filter \cite{volume_normalized_SPW}. The goal of this filter is to minimize the reconstruction error of the particle density. Each Fourier coefficient $(q_{hkl})$ of the unfiltered reconstruction $\bm{S}_{raw}$ is modulated as follow, to produce the 3D-CTF amplitude corrected filtered reconstruction, $\bm{S}$.

\begin{equation} \label{eq:wiener_filter}
    \bm{S}({q}_{hkl}) = 
        \mathcal{F}^{-1} \left\{ 
            \dfrac{ \mathcal{F} \left\{ \bm{S}_{raw} \right\}({q}_{hkl}) }{ \bm{W}_{thres}^{2}({q}_{hkl}) + {1}\ /\ {\bm{\mathrm{SNR}}({q}_{hkl})} }\ \bm{W}_{ad\text{-}hoc}({q}_{hkl})
        \right\}
\end{equation}
where:
\begin{itemize}
    % Sampling function
    \item $\bm{W}_{thres}$ is the total thresholded sampling function, i.e. the sampling function of the reference, calculated in section \ref{sec:algo:avg:subtomo_avg}.

    % SNR
    \item $\bm{\mathrm{SNR}}$ is the spectral Signal-to-Noise ratio estimate of $\bm{S}_{raw}$. It is defined as follow:
    \begin{equation} \label{eq:psnr}
    \begin{tikzpicture}[baseline=(g1.base),level/.style={},decoration={brace,mirror,amplitude=7}]

        \node (g1) {$\bm{\mathrm{SNR}}(q_{hkl}) =$};

        \node (g2) [right of=g1, xshift=2.3cm] {$\left(\dfrac{ 2\ \bm{\mathrm{FSC}}_{{\nabla}cut}(q_{hkl}) }{ 1-\bm{\mathrm{FSC}}_{{\nabla}cut}(q_{hkl}) }\right)$};

        \node (g3) [right of=g2, xshift=2.9cm] {$\left(\dfrac{1}{\mathrm{G} \otimes \bm{W}_{thres}^{2}(q_{hkl})} \right)$};
    
        % Annotations
        \draw [decorate] ($(g2.west)+(0.2,-0.6)$) --node[below=2.1mm]{$(1)$} ($(g2.east)+(-0.2,-0.6)$);
        % \draw [decorate] ($(g3.west)+(0.2,-0.6)$) --node[below=2mm]{$(2)$} ($(g3.east)+(-0.2,-0.6)$);
        \draw [decorate] ($(g3.west)+(0.2,-0.6)$) --node[below=2.1mm]{$(2)$} ($(g3.east)+(-0.2,-0.6)$);
    \end{tikzpicture}
    \end{equation}
    where:
    \begin{itemize}
        \item $(1)$ defines the map interpretability (i.e. the signal to noise ratio) found in the unfiltered data, including the noise found in the solvent region. As discussed in \cite{rosenthal_2003}, it is equal to the square of cosine of the average phase error and is equivalent to the square of the figure-of-merit (FOM) used in X-ray crystallography. It is described from section \ref{sec:algo:avg:molecular_mask} to \ref{sec:algo:avg:fom}.
        \item $(2)$ is a Gaussian smoothed version of the total sampling function. The FOM assumes that the number of Fourier measurements is homogeneous. This condition is not met with non-homogeneously spread subtomogram defoci and with strongly preferred orientations in the data set. To correct for these cases, each Fourier component of the FOM is weighted by the sum of squared CTF values (i.e. $\bm{W}_{thres}^{2}(\bm{q}_{hkl})$), which can be considered as the \textit{effective} number of Fourier measurements.
        % I am not sure about this explanation. I think it is because the FSCs are calculated on the CTF-corrected volumes (S_raw / W), but the SNR is meant to be applied, via the Wiener filter, on the raw (not CTF-corrected) volumes. So we have to multiply it by W. (S_raw * W / W = S_raw). Both explanations may be correct actually...
    \end{itemize}

    \item $\bm{W}_{ad\text{-}hoc}$ is an ad-hoc filter described in \ref{sec:algo:avg:wiener}. It contains the B-factor sharpening and the Modulation Transfer Function (MTF) of the detector.
\end{itemize}

\subsubsection{Mask and box size} \label{sec:algo:avg:box}

Originally calculated from the desired mask size (\code{Ali\_mRadius} and \code{Ali\_mType}), the box size also includes additional voxels to account for the mask roll-off, the padding factor and an eventual additional padding to optimize the final box size to Fourier transforms.

\subsubsection{Sampling functions} \label{sec:algo:avg:SF3D}

The 3D-sampling function (i.e. a weighted 3D-CTF model \cite{bharat_2015}) describes the extent of information transfer during the imaging process. It is supposed to track all of the systematic changes to the signal imposed by the microscope and image processing algorithms. As such, it should take into account the CTF modulations for each subtomogram in each image of the tilt-series, while accounting for the increasing sample thickness with tilt angle and the accumulated electron dose. Moreover, it should account for any filtering applied to the tilt-series or weighting applied during tomogram reconstruction.

{\emClarity} is currently calculating a simplified model of the sampling function of the particles, referred as $\bm{w}$, but a per-particle model, including every weights applied during the back-projection, will be added in the future. The current model groups particles from the same tilt-series by dividing the field of view into strips parallel to the tilt-axis. The number of strips, also referred as group, can be changed by the \code{ctfGroupNumber} parameter and is set by default to 9. During initialization (section \ref{sec:init}), each particle is assigned to the group it belongs to, thus, to a given sampling function $\bm{w}$. Although particles from a given group have the same $\bm{w}$, their contribution to the total sampling function $\bm{W}$, i.e. the sampling function of the reference, is different, because they have different orientations in the tomogram.

When reconstructing an image, the microscope multiplies the spectrum of the object by the CTF. Before the tomogram reconstruction, we multiplied again the images by the CTF to correct for the negative contrast (phase correction). At this point, the spectrum of the tomogram is multiplied by the square of CTF. Therefore, to correct for the CTF amplitudes, we will need to divide the reconstruction spectrum by CTF squared. As such, we calculate directly the square of the sampling functions, $\bm{w}^2$.

In conclusion, {\emClarity} first needs to calculate the sampling function $\bm{w}^2$ of the particles. To do so, a 3D super-sampled spectrum is filled with $i$ CTFs, once for each view $i$ of the tilt-series. Briefly, for each tilt-series $t$, for each group $g$ and for each view $i$:

\begin{enumerate}
    \item \textbf{Get the CTF}: Assuming the tilt-axis is parallel to the $y$ axis and assuming that the defocus is a positive number, we can calculate the defocus value at the center of the strip $g$, such as:
    \begin{equation}
        \bm{\mathrm{z}}_{t,g,i} = \bm{\mathrm{z}}_{t,i} - x_{g}\tan(\bm{\alpha}_{t,i})
    \end{equation}
    where $\bm{\mathrm{z}}_{t,i}$ is the defocus value calculated in section \ref{sec:algo:defocus_estimate:per_view_defocus}. $x_{g}$ is the center of the group $g$ in the $x$ direction. The $x$ coordinates are centered, meaning that $x=0$ at the tilt-axis. $\bm{\alpha}_{t,i}$ is the tilt-angle. Note that the further away in $x$ the group is from the tilt-axis, the more the defocus varies with the tilt-angle. We can then calculate the 2D CTF ${[\bm{w}^2]}_{t,g,i}$, as in equation \ref{eq:ctf2}, using this defocus value, the defocus astigmatic shift $\bm{\Delta \mathrm{z}}_{t,i}$ and the astigmatic angle $\bm{\phi}_{t,i}$, both calculated in section \ref{sec:algo:defocus_estimate:per_view_defocus}.

    \item \textbf{$\bm{z}$-stretch}: To take into account the thickness of the view, ${[\bm{w}^2]}_{t,g,i}$ is stretched in $z$, thus becoming a volume of 9 slices, with the total weight along the slices kept to 1.

    \item \textbf{Weighting}: The stretched ${[\bm{w}^2]}_{t,g,i}$ is ``fraction'' filtered in the same way the corresponding view $i$ in the tilt-series was. Indeed, during the alignment of the tilt-series (section \ref{sec:algo:defocus_estimate:transform}), we kept track of the inelastic scattering events by multiplying each view of the tilt-series by its fraction of inelastic ${[\bm{f}_{inelastic}]}_{t,i}$ (equation \ref{eq:f_inelatic}). As such, ${[\bm{w}^2]}_{t,g,i}$ is also multiplied by its ${[\bm{f}_{inelastic}]}_{t,i}$ and similarly by the fraction of dose, if a Saxton dose-scheme was used. Additionally, an exposure filter ${[\bm{W}_{exposure}]}_{t,i}$ was applied to the tilt-series before the tomogram reconstruction (section \ref{sec:algo:ctf_3d:ctf_phase_correction}) and is therefore applied to ${[\bm{w}^2]}_{t,g,i}$ as well.

    \item \textbf{From 2D to 3D}: The $z$-stretched and weighted ${[\bm{w}^2]}_{t,g,i}$ is then rotated by its corresponding tilt-angle $\bm{\alpha}_{t,i}$, using linear interpolation, and is added to the pre-allocated 3D spectrum $\bm{w}_{t,g}^2$. Once every ``view'' $i$ has been added, $\bm{w}_{t,g}^2$ is finally normalized between 0 and 1.
    \begin{note}By replacing the CTFs in the 3D space, the sampling functions ultimately include the so-called ``missing-wedge''.\end{note}
    % Each view of exposure filter is rotated as well and placed into another 3D spectrum, referred as $E_{i}$. To normalise each frequency of the $S_{i}$ between 0 and 1, it is finally divided by $E_{i}$.
    % This is not clear...
\end{enumerate}

As such, for each tilt-series $t$, we calculate $g$ sampling functions, all of which are saved into the same montage \code{<projectDir>/cache/<prefix>\_<bin>.wgt}. These functions are in the microscope frame and will be rotated into the reference frame to calculate the total sampling function $\bm{W}^2$.

\subsubsection{Subtomogram average and total sampling function} \label{sec:algo:avg:subtomo_avg}

This step produces the two ${[\bm{S}_{raw}]}_1$ and ${[\bm{S}_{raw}]}_2$, i.e the two subtomogram average, and the two ${[\bm{W}_{thres}]}_1$ and ${[\bm{W}_{thres}]}_2$, i.e. the two total sampling functions, from equation \ref{eq:wiener_filter}. This entire section is repeated for each one of the two half-set.

We first pre-allocate in memory $\bm{S}_{raw}$ and $\bm{W}^2$. The following steps consist into filling these empty volumes with every transformed subtomogram and every $\bm{w}^2$ weight, respectively. For each sub-regions registered into the metadata during \code{init}, we extract the sub-region coordinates, the particles information ($x,\ y,\ z$ coordinates and rotation) and for each particle $p$:
\begin{enumerate}
    \item \textbf{Particle extraction}: The subtomogram $\bm{s}_{p}$ is extracted from its sub-region tomogram. There are a few things worth to mention:
    \begin{enumerate}
        \item If a particle, defined by \code{particleRadius}, is out of the boundaries of its sub-region, it is ignored and will be ignore in the following cycles. The box size is often much larger than the particle, to ensure that all of the delocalized information is available for full CTF restoration. Ideally, this is fully represented in every particle, but as long as the particle itself is in the sub-region, we do not worry if the high resolution information is not there and extract the particle nonetheless.
        \item The subtomogram coordinates are floats and are likely to have decimals. To preserve the center of the particle, we take the nearest integers to extract the box from the sub-region tomogram and the remaining rounded decimals (i.e. shifts) ${[\bm{T}_{orig}]}_p$ will be applied during the subtomogram transformation, at the next step.
        % we actually floor() and not round() if I remember correctly.
    \end{enumerate}
    \item \textbf{Get the particle in the reference frame}: Once extracted, the particle $\bm{s}_{p}$ is rotated into the reference frame, shifted by their ${[\bm{T}_{orig}]}_p$ shift resulting from the extraction and the symmetry is applied, if any. Each symmetry pair is calculated and averaged with the other ones to produce the symmetrized rotated particle.

    \item \textbf{Get the sampling function in the reference frame}: The same rotation is applied to the sampling function $\bm{w}_{p}^2$ to correctly represent the original sampling of the subtomogram.

    \item \textbf{CCC Weighting}: If $\bm{s}_{p}$ has an alignment score $\bm{\mathrm{CCC}}_{p}$ below the average score of the entire data-set, $\overline{\bm{\mathrm{CCC}}}$, both of $\bm{s}_{p}$ and $\bm{w}_{p}^2$ $q_{hkl}$ Fourier coefficients are down-weighted by the following filter:
    \begin{gather}
        \Delta \bm{\mathrm{CCC}}_{p} = \cos^{-1}(\bm{\mathrm{CCC}}_{p}) - \cos^{-1}(\overline{\bm{\mathrm{CCC}}})\notag\\
        {[\bm{w}_{\mathrm{CCC}}]}_p(q_{hkl}) = \exp \left({ -\left(\code{flgQualityWeight} \times \Delta \bm{\mathrm{CCC}}_{p}\right)^2 q_{hkl}^2 }\right)
    \end{gather}
    The further away the particle is from the mean, the stronger the filter. It mostly affects high frequencies. This feature is deactivated for the first cycle as it requires CCC scores, which are calculated during the subtomogram alignment procedure (section \ref{sec:align}).

    \item \textbf{Build the reference}: Now that both $\bm{s}_{p}$ and $\bm{w}_{p}^2$ are in the reference frame, we can add them to the pre-allocated volumes $\bm{S}_{raw}$ and $\bm{W}_{p}^2$, respectively. There are a few things worth to mention:
    \begin{itemize}
        % I am not mentioning the experimental Bfactor based on defocus.

        \item The particle $\bm{s}_{p}$ is centered and standardized before being added to the reference. The corners of the box, which can contain non-sampled regions, are also ignored.

        \item Particles with an alignment score below \code{flgCCCcutoff} are ignored. By default, \code{flgCCCcutoff} is 0 (i.e. no particle ignored).

        \item If \code{flgCutOutVolumes} is 1, the transformed (rotated + shifted) particles are saved to disk in the \code{cache} directory. Before being saved, the subtomograms are padded by 20 pixels.
    \end{itemize}
\end{enumerate}

The reconstruction $\bm{S}_{raw}$ is centered, standardized and saved in the project directory as \code{cycleXXX\_<project>\_class0\_REF\_*\_NoWgt.mrc}.

As we will discussed in the next sections, our PSNR estimate directly comes from the Fourier Shell Correlation (FSC), making it less reliable for critically under-sampled frequencies. We can use the total sampling function $\bm{W}^2$, which weights the PSNR, to correct for this effect. To do so, frequencies below a given threshold are up-weighted, which effectively will down-weight these frequencies in the final reconstruction $\bm{S}$. This ``thresholded'' sampling function is referred as $\bm{W}_{thres}^2$ and it is calculated as follow:
\begin{enumerate}
    \item {\emClarity} defines critically under-sampled frequencies as the frequencies that have a weight less than $0.2 \times \mathrm{median}(\bm{W}^2)$, excluding frequencies with a value of 0 from the calculation of the median.
    \item The new sampling value of the frequencies below the threshold was defined empirically and is equal to:
\begin{lstlisting}
low = 0.1 * max(W2);
threshold = 1.5 * median(W2(W2 > low) - low);
\end{lstlisting}
    \item Frequencies at the ``edge'' of critically under-sampled regions are slightly up-weighted too, to create a smooth transition between this new threshold value and the rest of the sampling function.
    \item This total thresholded sampling function $\bm{W}_{thres}^2$ is saved in the project directory as \code{cycleXXX\_<project>\_class0\_REF\_*\_Wgt.mrc}.
\end{enumerate}

We now have calculated the raw subtomogram average $\bm{S}_{raw}$ and the total sampling function $\bm{W}_{thres}^2$, of equation \ref{eq:wiener_filter}. As such, we can already correct for all of the systematic changes in $\bm{S}_{raw}$ imposed by the microscope and image processing algorithms, such as:
\begin{equation}
    \bm{S}_{corr} = \mathcal{F}^{-1} \left\{ \frac{ \mathcal{F} \{ \bm{S}_{raw} \} }{ \bm{W}_{thres}^2 } \right\}
\end{equation}
This correction does not take the SNR of the reconstructions into account and even if the total sampling function is thresholded, the noise present in $\bm{S}_{raw}$ is amplified in $\bm{S}_{corr}$. The objective of the next few sections consists into finding the best SNR estimate of the $\bm{S}_{corr}$ half-maps and apply a new correction that does take the SNR into account.

\subsubsection{Molecular masks} \label{sec:algo:avg:molecular_mask}

The SNR can be most accurately estimated from a ``masked'' FSC, calculated from the two $\bm{S}_{corr}$ half-maps, where the environs solvent noise is suppressed with soft-edged masks. To avoid introducing spurious correlations between half-sets, two masks are created, one for each half-map, as described below:
\begin{enumerate}
    \item The CTF-corrected reconstruction, $\bm{S}_{corr}$, is median filtered to remove ``salt and pepper'' noise and low-pass filtered to 14\r{A}. The volume is centered and standardized, and the edges of the reconstruction are forced to smoothly go to zero to suppress edge artifacts.
    \item The highest intensity voxels are then selected and used as seeds for an iterative dilation. The dilation progressively dilates the seeds toward their neighbour voxels that are above a gradually relaxed intensity threshold. This procedure ensure a connectivity-based expansion.
    \item The binary mask formed by the selected voxels after dilation, $\bm{M}_{bin}$, is Gaussian blurred and saved as $\bm{M}_{core}$. This mask is considered as the particle volume (non-hydrated), leaving only the strongest part of the densities.
    \item Additionally, $\bm{M}_{bin}$ is radially expanded by at least 10\r{A}, Gaussian blurred and saved as $\bm{M}_{mol}$. At this point, the $\bm{M}_{mol}$ mask includes the particle density \emph{and} its surrounding solvent to make sure flexible densities are correctly included in the FSC calculation, thereby preventing overestimation of the resolution.
    \item As we'll see, the two $\bm{S}_{corr}$ spectra are masked by their corresponding $\bm{M}_{mol}$ mask before computing the FSC. As such, we will need to ``scale'' the FSC to compensate for the masked-out voxels and the solvent content. The ratio $\bm{f}_{mask} / \bm{f}_{particle}$ does exactly that and is applied to the FSC at the next section, in equation \ref{eq:scale_FSC}.
    \begin{itemize}
        \item $\bm{f}_{mask}$ is the fractional volume allowed by the $\bm{M}_{mol}$ mask.
        \item $\bm{f}_{particle}$ is the fractional volume occupied by the particle within the box size. It is the volume allowed by the $\bm{M}_{bin}$ mask.
        \item As $\bm{M}_{mol}$ is a soft-edged mask and therefore contains values between 0 and 1, and even though the edges almost certainly contain surrounding solvent, we estimate the signal reduction in the soft-edged regions of the mask as they could cut through actual densities. This power reduction is implicitly included in $\bm{f}_{particle}$.
    \end{itemize}
    As we produce two $\bm{f}_{mask} / \bm{f}_{particle}$ ratio (one for each $\bm{S}_{corr}$), we take the average of both ratio.
\end{enumerate}

\subsubsection{Fourier Shell Correlation} \label{sec:algo:avg:fsc}

Before calculating the FSC between the two half-maps ${[\bm{S}_{corr}]}_1$ and ${[\bm{S}_{corr}]}_2$, we need to pre-process them:
\begin{enumerate}
    \item The \code{Ali\_mType} and \code{Ali\_mRadius} parameters are used to compute soft-edged shape mask, $\bm{M}_{shape}$. In order to constrain the analysis to the desired region, this mask is applied to the molecular masks $\bm{M}_{core}$ and $\bm{M}_{mol}$. The soft-edged part of $\bm{M}_{shape}$ (i.e. the taper) is randomized to reduce the correlation due to the masking during FSC calculation. 
    
    \item The reconstructions need to be aligned to each other. To do so, {\emClarity} uses the ``Fit in map'' procedure of {\Chimera}. As we want to base the alignment on the core of the particle, the averages are temporally masked with $\bm{M}_{core}$ before calling {\Chimera}. The two masked volumes are saved as \code{FSC/cycleXXX\_<prefix>\_Raw-*Ali.mrc}, then {\emClarity} extracts the output rotation $\bm{R}_{gold}$ and translation $\bm{T}_{gold}$, and use them to align ${[\bm{S}_{corr}]}_1$ to ${[\bm{S}_{corr}]}_2$, using spline interpolation.
    
    \item The two reconstructions are centered, standardized, and masked with their respective $\bm{M}_{mol}$ mask. Both reconstructions are saved as \code{FSC/fscTmp\_1\_noFilt\_*.mrc}. These volumes are padded to a cube of at least 384 voxels of length, to oversample their Fourier spectrum before the FSC calculation.
\end{enumerate}

Once the reconstructions are aligned and masked, we can calculate the spherical FSC:
\begin{enumerate}
    \item The two aligned and masked volumes are Fourier transformed and the cross-correlation spectrum is calculated:
    \begin{equation}  \label{eq:fsc_cc}
        \bm{\mathrm{CC}}(q_{hkl}) = \mathcal{F} \left\{ {[\bm{S}_{corr}]}_1 \right\} (q_{hkl})\ \overline{ \mathcal{F} \left\{ {[\bm{S}_{corr}]}_2 \right\} }(q_{hkl})
    \end{equation}
    where $\bar{x}$ is the conjugate of $x$.
    \item Each axis is divided into shells. The number of shells depends on the size of the Fourier spectrum. As the reconstructions are cubic, each shell is an empty spherical mask with a given thickness and delimits the set of $\bm{\mathrm{CC}}$ scores that will be averaged together. The final FSC curve is nothing more than the concatenation of the normalized $\bm{\mathrm{CC}}$ scores, such as:
    \begin{equation} \label{eq:fsc}
        \bm{\mathrm{FSC}}(shell) = \frac{ \sum\limits_{q=h',k',l'}^{Q'} \bm{\mathrm{CC}}(q_{h'k'l'}) }{ 
                                          \sqrt{ \sum\limits_{q=h',k',l'}^{Q'} {|\mathcal{F} \left\{ {[\bm{S}_{corr}]}_1 \right\}|}^2 \times
                                                 \sum\limits_{q=h',k',l'}^{Q'} {|\mathcal{F}\left\{ {[\bm{S}_{corr}]}_2 \right\}|}^2 } }
    \end{equation}
    where the $q_{h'k'l'}$ frequencies are the $q_{hkl}$ frequencies that belongs to the given shell. $\bm{Q}'$ is the number of $q_{h'k'l'}$ frequencies within each shell. The numerator is the cross-correlation score of the shell. The denominator is the ``power'' of the shell or in other words, the total complex magnitude of the shell. $|x|$ is the complex magnitude of the complex number $x = a+bi$, such as $|x| = \sqrt{a^2 + b^2}$. Note that $|\bar{x}| = |x|$.
\end{enumerate}

Moreover, we can calculate the FSC over conical shells, as opposed to spherical shells, making the SNR estimate via the FSC much more robust to resolution anisotropy \cite{conical_fsc}. To do so, {\emClarity} subdivide each spherical shell into 37 overlapping cones of 36\textdegree\ of half angle, with a 30\textdegree\ increment between each cone. Then, it recalculates the FSC as described above, but for each cone. We refer to these FSCs as $\bm{\mathrm{FSC}}_{\nabla}$, where as opposed to equation \ref{eq:fsc}, the normalized cross-correlation is calculated over the $q_{h''k''l''}$ frequencies, which are the frequencies within the current shell \emph{and} within the current cone. Similarly, $\bm{Q}''$ is the number of $q_{h''k''l''}$ frequencies within each shell of each cone.

Then, the spherical FSC, as well as the 37 conical FSCs, needs to be scaled by $\bm{f}_{mask}$ and $\bm{f}_{particle}$. Indeed, as discussed in section \ref{sec:algo:avg:molecular_mask}, these FSCs are calculated from masked volumes, where the object of interest represents only a fraction of the reconstruction. As such, each shell, whether spherical or conical, of each FSC, is scaled as follow. This scaling is applied up to the first shell that shows no correlation between the two half-maps (i.e. $\bm{\mathrm{FSC}}(shell)\leqslant0$).
\begin{equation} \label{eq:scale_FSC}
    \bm{\mathrm{FSC}}(shell) = \dfrac{ \bm{\mathrm{FSC}}(shell) \times (\bm{f}_{particle}/\bm{f}_{mask})}{1 + (1 - \bm{f}_{particle}/\bm{f}_{mask}) \times |\bm{\mathrm{FSC}}(shell)|}
\end{equation}

So far, the FSCs have been a function of shells, i.e $\bm{\mathrm{FSC}}(shell)$. This is also true for $\bm{Q}'$ and $\bm{Q}''$, which can be expressed as a function of spherical or conical shells, respectively. The next sections of this chapter are using an oversampled version of these curves. The oversampled curves are calculated using spline interpolation over 501 $h$ frequency points.

\subsubsection{Frequency cutoffs}  \label{sec:algo:avg:res_cutoffs}

The FSC directly impacts our SNR estimate, thereby our filtering. Defining a FSC threshold level at which the resolution is reproducible is therefore necessary. The one-bit and half-bit information curves indicate the resolution level at which enough information has been collected for interpretation and will be used to define the frequency cutoffs for filtering the reference.

The one-bit and half-bit curves, $\bm{T}_{1\text{-}bit}$ and $\bm{T}_{1/2\text{-}bit}$ respectively, are calculated for the spherical FSC and each of the conical FSCs, as in \cite{fsc_mvh}:
\begin{equation} \label{eq:n_effective}
    \bm{Q}_{e}(h) = \bm{Q}(h)\ /\ (2 \times U)
\end{equation}
\begin{equation}
    \bm{T}_{1\text{-}bit}(h) = \dfrac{ 0.5 + 2.4142\ / \sqrt{\bm{Q}_{e}(h)} } { 1.5 + 1.4142\ / \sqrt{\bm{Q}_{e}(h)} }
\end{equation}
\begin{equation}
    \bm{T}_{1/2\text{-}bit}(h) = \dfrac{ 0.2077 + 1.9102\ / \sqrt{\bm{Q}_{e}(h)} } { 1.2071 + 0.9102\ / \sqrt{\bm{Q}_{e}(h)} }
\end{equation}

As discussed in \cite{fsc_mvh}, the FSC is influenced with various factors, notably the number of voxels $\bm{Q}$ in a given Fourier shell, the number of repeated unit $U$ in the reconstruction and the size of the particle, all of which are taken into account in $\bm{Q}_{e}$ (equation \ref{eq:n_effective}). Of course, for the spherical FSC, $\bm{Q}$ corresponds to $\bm{Q'}$, whereas for the conical FSCs, it corresponds to their respective $\bm{Q''}$.

For each conical FSCs, we then define a frequency cutoff $\bm{h}_{cut}$, where $\bm{h}_{1\text{-}bit}$ and $\bm{h}_{1/2\text{-}bit}$ are the frequencies at which the FSC first goes below the $\bm{T}_{1\text{-}bit}$ or $\bm{T}_{1/2\text{-}bit}$ curves, respectively.
\begin{equation} \label{eq:hcut}
    \bm{h}_{cut} = (\bm{h}_{1\text{-}bit} - \bm{h}_{1/2\text{-}bit}) / 2
\end{equation}

The frequency cutoff is also calculated for the spherical FSC, as well as the fixed-valued threshold $\bm{h}_{0.5}$ and $\bm{h}_{0.143}$. These last two are only used for display and are saved in \code{FSC/cycleXXX\_<project>\_Raw-1-fsc\_GLD.pdf} or in the corresponding \code{.txt} file.

\subsubsection{Figure-Of-Merit} \label{sec:algo:avg:fom}

As discussed in \cite{rosenthal_2003}, the figure-of-merit (FOM) in equation \ref{eq:psnr}, also referred as $\bm{C}_{ref}$, defines the resolution criterion that will be used by the Wiener filter as SNR estimate. It is worth noting that ``$\bm{C}_{ref}$ is equal to the cosine of the average phase error and is equivalent to the crystallographic figure-of-merit, a common measure of map interpretability in X-ray crystallography'' \cite{rosenthal_2003}. $\bm{C}_{ref}$ is defined as follow:
\begin{equation} \label{eq:cref}
    \bm{C}_{ref}(h) = \sqrt{\dfrac{2\ \bm{\mathrm{FSC}}(h)}{1+\bm{\mathrm{FSC}}(h)}}
\end{equation}

To be used by the Wiener filter, $\bm{C}_{ref}$ needs to be expressed as a function of $q_{hkl}$ frequencies, i.e. it needs to be a volume of the same size as the spectrum about to be filtered. Additionally, $\bm{C}_{ref}$ should be limited by the frequency cutoffs computed in the previous section.

\begin{enumerate}
    \item Each one of the 37 conical $\bm{\mathrm{FSC}}_{\nabla}$ are multiplied with their corresponding frequency cutoff curve. These curves are equal to 1 (no weighting), up to their frequency cutoff $\bm{h}_{cut}$ where an exponential decay starts and forces the FSC to rapidly drop to zero. The cut-offed conical FSCs are referred as $\bm{\mathrm{FSC}}_{\nabla cut}$.
    % gaussian roll-off vs exponential decay... don't know.

    \item To express $\bm{C}_{ref}$ as a function of $q_{hkl}$ frequencies, {\emClarity} computes the ``anisotropic 3D FSC''. To do so, the 37 overlapping and oversampled conical $\bm{\mathrm{FSC}}_{\nabla cut}$ are reshaped and expanded back into their corresponding cone in the 3D space, while taking into account that frequencies overlapped with multiple cones are appropriately down-scaled.
\end{enumerate}

Once $\bm{\mathrm{FSC}}_{\nabla cut}(q_{hkl})$ is calculated, we can define $\bm{C}_{ref}(q_{hkl})$ as shown below. In practice, as with the total sampling function, we calculate directly $\bm{C}_{ref}^2$.
\begin{equation} \label{eq:cref3d}
    \bm{C}_{ref}(q_{hkl}) = \sqrt{\frac{ 2\ \bm{\mathrm{FSC}}_{\nabla cut}(q_{hkl}) }{ 1 + \bm{\mathrm{FSC}}_{\nabla cut}(q_{hkl}) }}
\end{equation}

\subsubsection{Volume-normalized SPA Wiener filter} \label{sec:algo:avg:wiener}

At this point, we have described every component of equation \ref{eq:wiener_filter}, expect the $\bm{W}_{ad\text{-}hoc}$ weighting function. This function should be thought as an external component of the Wiener filter; it is just an additional ad-hoc filtering that we apply on the final volume. $\bm{W}_{ad\text{-}hoc}$ contains two components, the global B-factor sharpening and the default Modulation Transfer Function (MTF) of the detector.
\begin{equation}
    \bm{B}_{factor}(q_{hkl}) = \exp\left( \frac{ \code{Fsc\_bfactor} \times q_{hkl}^{2} }{4} \right)
\end{equation}
\begin{equation} \label{eq:mtf}
    \bm{\mathrm{MTF}}(q_{hkl}) = \frac{0.13}{ \exp \left( -20\ q_{hkl}^{1.25} \right) + 0.13 }
\end{equation}

In equation \ref{eq:wiener_filter}, we clearly show that we divide each Fourier coefficient by some weight, which depends on the sampling and on our SNR estimate. As we have now seen, the SNR estimate is calculated using the CTF-corrected $\bm{S}_{corr}$ half-maps. Here's a reformulation of equation \ref{eq:wiener_filter} with $\bm{S}_{corr}$, as opposed to $\bm{S}_{raw}$:
\begin{equation} \label{eq:wiener_filter2}
    \bm{\mathrm{S}}({q}_{hkl}) = 
        \mathcal{F}^{-1} \left\{ 
            \dfrac{ { \bm{W}_{thres}({q}_{hkl}) }^2 }{ { \bm{W}_{thres}({q}_{hkl}) }^2 + {1}\ /\ {\bm{\mathrm{SNR}}({q}_{hkl})} }\ \mathcal{F} \left\{ \bm{S}_{corr} \right\}\!({q}_{hkl})\ \bm{W}_{ad\text{-}hoc}({q}_{hkl})
        \right\}
\end{equation}

% adHocMTF = 1 / (exp(-20 * q^1.25) + 0.13) / 0.13)
% bF = exp(bFactor * x^2 / 4)
% >> adHocMTF * bF * forceMaskAli

% WEIGHT
% FOM = 2 * anisoFSC / (1-anisoFSC)
% weight = bFactor.* 1/(wgt + (1/(snrWeight * FOM) .* avgCTF ) .* nyquistLimit;
% weight * REF.



%%%%%%%%%%%% END


%%%%% WORKFLOW:
% CTF CORRECTION: load the REF, threshold SF3D and divide by it.
% MASKS:
%   1) Prepare randomized shape masks (VOLMASK) and nyquist lowpass (BANDPASSFILT).
%   2) Compute the FSC masks (SHAPEMASK and PV) and particle fraction as well (take the mean).
%      Both masks are multiplied by the randomized shape mask VOLMASK.
% ALIGN VOLUMES:
%   1) Apply the randomized PV to the REFs and save to disk for chimera BUT don't modify REFs.
%   2) Fit both map with chimera (rotation+shift). Apply the transformation (spline) to align the REFs. REF1 is modified here.

% PREPARE VOLUME
%   1) Apply BANDPASSFILT, SHAPEMASK *and* VOLMASK to REF and set mean and variance. Save to REF_FILT.
%   2) REF_FILT are saved  as fscTmp_*_EVE/ODD.mrc.

%%%%%%%%%%%%%%%%
%% RANDOM FSC
%%   1) Calculate FSC cutoff (make the cutoff between two shells).
%%      The lowcut is about 1/(3*currentResForDefocusError) - ~nyquist.
%%   2) Randomize the phases of both REF after the cutoff and pad to at least 384. Save to REF_RAND.
%%      At this point, REF is not masked.
%%   3) Apply PV to REF_RAND
%%   4) Calculate fft of REF_RAND and calculate the FSC (only spherical): FSC_RAND.
%% NORMAL FSC
%%   1) Set mean, variance and apply SHAPEMASK to REF. REF are modified here.
%%   2) Save REF to fscTmp_*_noFilt_EVE/ODD.mrc.
%%   3) Pad REF to at least 384 and calculate the conical FSC: FSC_NORMAL. REF are modified here (padding).
%% TIGHT FSC
%%   1) Calculate fft of PV*REF and calculate the conimal FSC: FSC_TIGHT. REF are NOT modified here.
%%   2) Fit a curve to FSC_TIGHT: FSC_TIGHT_FIT
%%%%%%%%%%%%%%%%

% SCALE FSC_NORMAL
%   1) Scale the FSC_NORMAL up to the first 0. The scaling is: (f_particle * FSC_NORMAL) / (1 + (f_particle-1) * abs(FSC_NORMAL)).
% FITTING FSCs
%   1) Fit a curve: FSC_NORMAL_FIT.
%   2) Do the same but with the FSC_NORMAL_NUM: FSC_NORMAL_NUM_FIT.

%%%%%%%%%%%%%%%%
%% RANDOM FSC
%%   1) Fit curve to FSC_RAND: FSC_RAND_FIT.
%%   2) FSC_TRUE = FSC_TIGHT_FIT up to cutoff, then subtract* FSC_RAND_FIT to FSC_TIGHT_FIT.
%%      Actually, (FSC_TIGHT_FIT - FSC_RAND_FIT) / (1-FSC_RAND_FIT)
%%      Everything is oversampled to 501 points.
%%   3) Save PV to *-particleVolEst_*.mrc and SHAPEMASK to *-shapeMask_*.mrc
%%%%%%%%%%%%%%%%

% FSC CUTOFFS
%   1) Compute the curves
%      ONEBIT = (0.5 + 2.4142 / sqrt(nEffective)) / (1.5 + 1.4142 / sqrt(nEffective));
%      HALFBIT = (0.2077 + 1.9102 / sqrt(nEffective)) / (1.2071 + 0.9102 / sqrt(nEffective));
%      with nEffective = FSC_NORMAL_NUM_FIT * (3/2 * DbyL)^2 / (2*SYMMETRY); FSC_NORMAL_NUM_FIT is oversampled to 501 points. DbyL=2/3
%      ALIBIT = mean(ONEBIT, HALFBIT).
%   2) Compute the cutoffs
%      ONEBIT_CUT  = find(FSC_NORMAL_FIT - ALIBIT < 0  & osX > 1/100, 1, 'first');
%      HALFBIT_CUT = find(FSC_NORMAL_FIT - HALFBIT < 0 & osX > 1/100, 1, 'first');
%      GLD_CUT     = find(FSC_NORMAL_FIT <= 0.143      & osX > 1/100, 1, 'first');
%      If not found, set the cuts to first frequency where pixelSize > 0.425.
%      This is done for every cone too.
%   3) FORCEMASK: Set this to 1 up to HALFBIT_CUT and then exp(-0.005^-2 * (osX-osX(HALFBIT_CUT).^2);
%      For the spherical, use HALFBIT_CUT, for the cones, use GLD_CUT.
%      Basically a curve that starts to decrease as the HALFBIT_CUT, up to 0.
%   4) FORCEMASK_ALI: Same thing, but start to decrease at ONEBIT_CUT.

% cREF/FOM
%   1) cREF = sqrt( abs(2.*FSC_NORMAL_FIT / (1 + FSC_NORMAL_FIT)) ) * forceMask;
%      This is oversampled to 501 points and done for each cone and cREF is normalized

% SAVE METADATA
% mRAW = masterTM.(cycle).fitFSC.Raw1 = {shellsFreq,
%                                        shellsFSC,
%                                        {cRef,cRefAli,bh_global_MTF},
%                                        osX,
%                                        forceMaskAlign,
%                                        forceMask,
%                                        nCones,
%                                        coneList,
%                                        halfAngle,
%                                        samplingRate};
% mRESAMPLE = masterTM.(cycle).(fitFSC).(ResampleRaw1) = [bestAnglesTotal(1:9); bestAnglesTotal(10:12),0,0,0,0,0,0];
% mMASK = masterTM.(cycle).(fitFSC).(MaskRaw1) = {maskType, sizeMask, maskRadius, maskCenter, 1};
% masterTM.(currentResForDefocusError) = 1/osX(oneBitCut);

% PLOT
%   1) cycleXXX_project_Raw-1-fsc_GLD.txt: table with spherical and conical FSC, oversampled.
%   2) cycleXXX_project_Raw-1-fsc_GLD.pdf: save to FSC, with ONE/HAlFBIT curves and threshold.
%   3) cycleXXX_project_Raw-1-cRef_GLD.pdf
%   4) cycleXXX_project_Raw-1-cRefAli_GLD.pdf
%

\newpage

\subsection{Subtomogram alignment} \label{sec:algo:align}

This goal of this procedure is to find the best transformation (rotation $+$ translation) between each particle and its reference. By convention, the rotation $\bm{R}$ and translation $\bm{T}$ are meant to be applied to the particles to align them into the reference frame.

% Subtomogram alignment is usually performed by angular search. Usually, the reference is masked in real space, rotated and weighted in Fourier space by a missing wedge mask or a sampling function. Then, the 3D Cross-Correlation (CC) is calculated. The position and value at each point in the 3D CC volume describes the similarity of the two volumes. The peak intensity defines the CC score and its position provides the shift.

% This description, often used to describe subtomogram averaging and alignment, is \textit{not} what {\emClarity} is doing. We will describe in detail in the next sections how the alignment procedure in {\emClarity} works, but it is worth presenting the main ideas now to see the global picture:

% The alignment is divided into three main step. First, the reference is aligned to the subtomograms and the position (i.e. translation) of the strongest peak is extracted. Then, this translation estimate is used to align the subtomograms to the reference and the CCC score is calculated. The rotation that gives the highest CCC score is selected. Finally, the reference is aligned once again to the subtomograms, but this time using only the best rotation (and the first translation estimate) and the position of the strongest peak is extracted. This gives us the final translation. The reason why we do this back and forth is mainly because it allows us to consider the particle symmetry are aligning the subtomogram to the reference frame.

\subsubsection{Masks and filters}

The box size is defined as in section \ref{sec:algo:avg:box} and every volume (particles, references, filters and masks) will be padded/cropped to this size.
\begin{itemize}
    \item \textbf{Masks}: Every voxel outside of the alignment mask $\bm{M_{ali}}$ is ignored throughout the alignment procedure. $\bm{M_{ali}}$ is defined by the \code{Ali\_mType}, \code{Ali\_mRadius} and \code{Ali\_mCenter} parameters. $\bm{M_{particle}}$ is an ellipsoid of radius equal to \code{particleRadius} and sets the maximum translation allowed, thereby preventing the particles to drift too much from their original position. $\bm{M_{peak}}$ is an optional rectangular mask of radius \code{Peak\_mRadius} used to further restrict the translations.
    
    \item \textbf{Filters}: To ignore the information after the current frequency cutoff, the subtomograms are going to be filtered before comparison with the reference. This filter, corresponds to the $\bm{C_{ref}}$ described in equation \ref{eq:cref3d}, but in this case, only the spherical shells and the frequency cutoff $\bm{h_{cut}}$ from the spherical FSC are used, whereas the conical FSCs were used during the subtomogram averaging. Additionally, the ad-hoc Modulation Transfer Function ($\bm{\mathrm{MTF}}$, equation \ref{eq:mtf}) will also be applied before comparison with the reference.
    
    \item \textbf{Sampling functions}: The sampling functions of the particles $\bm{w}$ should be already calculated and saved (section \ref{sec:algo:avg:SF3D}). If the box size changed between the averaging and the alignment, the sampling functions are calculated again. The sampling functions are created and saved in the microscope frame, thus we'll refer to them as $\bm{w}_{mic}$. The total sampling functions (i.e. the sampling function of the two half-maps) $\bm{W}$ are created and saved in the reference frame, thus we will refer referred to them as $\bm{W}_{ref}$.
\end{itemize}

\subsubsection{References}

The references (i.e. the two half-maps) are calculated and saved in the project directory during subtomogram averaging (section \ref{sec:algo:avg}) and are in the reference frame. As such, we will refer to them as $\bm{S}_{ref}$. These are CTF-corrected and filtered. Before aligning the particles to these reconstructions, some pre-processing is done:
\begin{enumerate}
    \item \textbf{Center-Of-Mass}: The references $\bm{S}_{ref}$ are masked with $\bm{M}_{ali}$ and their $\bm{M}_{core}$ mask (section \ref{sec:algo:avg:molecular_mask}) is calculated and applied. Then, the $x,\ y,\ z$ center-of-mass (COM) of the two masked $\bm{S}_{ref}$ are calculated and applied to have the center of the box aligned with the COM of the particle averages.

    \item \textbf{Combine low-resolution information}: We assume that the information from the DC-component (i.e. the zero frequency) to the initial resolution cutoff $\bm{h}_{cut}$ are ``shared'' between the two half-maps. As such, this ``shared'' information is averaged and the ``shared'' frequencies of both $\bm{S}_{ref}$ are replaced by this average. The initial resolution cutoff is set to 40\r{A} by default.
    % maybe explain a bit more where this 40A comes from.
\end{enumerate}

\subsubsection{Calculate the translation between a particle and a reference} \label{sec:algo:align:get_translation}

This section can be seen as a function that takes as inputs a subtomogram, $\bm{s}_{mic}$, which is in the microscope frame, a reference, $\bm{S}_{ref}$, which is in the reference frame, the particle's rotation $\bm{R}$ and translation $\bm{T}$. If the reference and the particle are both in the same frame, how do we calculate the translation $\bm{T}_{S\text{-}s}$ between them?

\begin{enumerate}
    \item \textbf{Get the reference into the microscope frame}: The reference $\bm{S}_{ref}$ is rotated by $\bm{R}^{T}$, translated by $-\bm{T}$, centered and standardized. To emphasize that the reference is now in the microscope frame, we will refer to it as $\bm{S}_{mic}$ from now on.
    \begin{note}\label{note:ref2mic_frame}By convention, $\bm{T}$ and $\bm{R}$ are meant to be applied to the particles and will switch the particles from the microscope to the reference frame. As such, if we want to align the reference to the microscope frame, we must apply the ``inverse`` transformation, such as $-\bm{T}$ and $\bm{R}^{T}$. In this case, the translations are applied after the rotation.
    \end{note}
    
    \item \textbf{Get the total sampling function into the microscope frame}: The total sampling function $\bm{W}_{ref}$ is rotated by $\bm{R}^{T}$ as well. To emphasize that the total sampling function is now in the microscope frame, we will refer to it as $\bm{W}_{mic}$ from now on.
    
    % C_Ref uses the spherical FSC because we're in the microscope frame! Using the conical would've meant rotating C_ref as well.
    \item \textbf{Calculate the CCC map}: The particle $\bm{s}^{mic}$ is masked with $\bm{M}_{particle}$, filtered with $\bm{C}_{ref}$ and $\bm{\mathrm{MTF}}$, and finally centered and standardized. Then, the CCC map is calculated as follow:
    \begin{equation}
        \bm{\mathrm{CCC}}_{map} = 
        \mathcal{F}^{-1} \left\{
        \mathcal{F} \{ \bm{s}_{mic} \}\
        \bm{W}_{mic}\
        \overline{ \mathcal{F} \{ \bm{S}_{mic} \}} \
        \bm{w}_{mic}
        \right\}
    \end{equation}

    \item \textbf{Further restrict the allowed translation}: If $\bm{M}_{peak}$ is defined, apply it to $\bm{\mathrm{CCC}}_{map}$. This effectively removes every peak outside the allowed region, set by \code{Peak\_mRadius}.
    % I am not sure this is very useful now. M_particle should already do the job.

    \item \textbf{Get highest peak position}: Extract the $x,y,z$ coordinates of the strongest peak, $\bm{p}_{x,y,z}$. In this case, the origin $(x=y=z=0)$ corresponds to the center of the CCC map. A $7\!\times\!7\!\times\!7$ box is extracted around the peak and we calculate the COM of this box, $\bm{\mathrm{COM}}_{x,y,z}$. The translation $\bm{T}_{S\text{-}s}$ is then defined as $\bm{T}_{S\text{-}s} = \bm{p}_{x,y,z} + \bm{\mathrm{COM}}_{x,y,z}$.
\end{enumerate}

\subsubsection{Calculate the CCC score between a particle and a reference} \label{sec:algo:align:get_CCC}

This section can be seen as a function that takes as inputs a particle, $\bm{s}_{mic}$, which is in the microscope frame, a reference, $\bm{S}_{ref}$, which is in the reference frame, the particle's rotation $\bm{R}$ and translation $\bm{T}$. If the particle and the reference are both in the same frame, how do we calculate their Constrained Cross-Correlation score, $\bm{\mathrm{CCC}}_{score}$?

\begin{enumerate}
    \item \textbf{Get the particle into the reference frame}: The particle is translated by $\bm{T}$ and rotated by $\bm{R}$. To increase the SNR of the transformed particle, we also apply its symmetry. To emphasize that the particle is now in the reference frame, we will refer to it as $\bm{s}_{ref}$ from now on.

    \item \textbf{Get the sampling function of the particle into the reference frame}: The same $\bm{R}$ rotation is applied to the sampling function of the particle to correctly represent the original sampling of the subtomogram. To emphasize that the sampling function is now in the reference frame, we will refer to it as $\bm{w}_{ref}$ from now on.
    \begin{note}As explained in section \myref{sec:algo:avg:SF3D}, {\emClarity} is currently not computing a sampling function per particle, but divides the field of view by 9 strips parallel to the tilt-axis.
    \end{note}
    % apply sort of radial weighting depending on the sampling (1/sqrt(bin)).

    \item \textbf{Calculate the CCC score}: The particle $\bm{s}_{ref}$ is masked with $\bm{M}_{ali}$, filtered with $\bm{C}_{ref}$ and $\bm{\mathrm{MTF}}$, and finally centered and standardized. The reference $\bm{S}^{ref}$ is centered and standardized as well. Then, the CCC score is calculated, as follow:
    \begin{equation}
        \bm{\mathrm{CCC}}_{score} = \frac{
        \sum\limits_{q=h,k,l}^{Q}
            \Re\left(
            \mathcal{F} {\{ \bm{s}_{ref} \}}_{q}\
            {[\bm{W}_{ref}]}_{q}\
            \overline{\mathcal{F} {\{ \bm{S}_{ref} \}}}_{q}\
            {[\bm{w}_{ref}]}_q
            \right)
        }{
        \sqrt{
            \sum\limits_{q=h,k,l}^{Q}
                { \abs[\Big]{ \mathcal{F} {\{ \bm{s}_{ref} \}}_{q}\ {[\bm{W}_{ref}]}_{q} } }^2
            \times
            \sum\limits_{q=h,k,l}^{Q}
                { \abs[\Big]{ \overline{\mathcal{F} {\{ \bm{S}_{ref} \}}}_{q}\ {[\bm{w}_{ref}]}_q } }^2
                }
        }
    \end{equation} % why Re() and not abs()?
    where $\mathcal{F}\{x\}$ is the fast Fourier transform of $x$. $\Re(x)=a$, where $x$ is a complex number such as $x=a+bi$. $|x|$ is the complex magnitude of the complex number $x$ and is defined as $\sqrt{a^2+b^2}$. $\overline{x}$ is the complex conjugate of $x$.
\end{enumerate}

\begin{note}Notice that when calculating the CCC scores, the particle is masked with $\bm{M}_{ali}$, whereas the particle is masked with $\bm{M}_{particle}$ when calculating the translation (step 3. section \myref{sec:algo:align:get_translation}). This allows to restrict the translation within the particle, while allowing the surrounding environment (neighbouring particles in a lattice, etc.) to contribute to the alignment.
\end{note}

\subsubsection{Evaluate the initial alignment} \label{sec:algo:align:eval_init_alignment}

First, we calculate the initial CCC score, $\bm{\mathrm{CCC}}_{init}$, of each particle $\bm{s}$. We do as follow, for each particle $p$:

\begin{enumerate}
    \item \textbf{Extraction}: The particle $\bm{s}_{p}$ is extracted it from its sub-region tomogram. This step is identical to step 3. from section \ref{sec:algo:avg:subtomo_avg}. As the extraction is in pixel space, any remaining decimals ${[\bm{T}_{orig}]}_p$ from its $x,y,z$ coordinates are saved and will be applied during the transformation, in step 3.
    
    \item \textbf{Get the initial rotation}: At this point of the alignment, the rotation of the particle comes from the previous cycle or directly from the picking if it is the first alignment. As such, the rotation is equal to ${[\bm{R}_{init}]}_p = \bm{R}_{n-1,p}$, where $n$ is the current cycle number.
    
    \item \textbf{Get the initial translation}: As the particle was just extracted, the only translation attached to the particle is the shifts ${[\bm{T}_{orig}]}_p$ left from the extraction. As such, the translation is equal to ${[\bm{T}_{init}]}_p = {[\bm{T}_{orig}]}_p$.
    
    \item \textbf{Get the CCC score}: Using the ${[\bm{R}_{init}]}_{p}$ and ${[\bm{T}_{init}]}_{p}$ defined above, calculate ${[\bm{\mathrm{CCC}}_{init}]}_p$, as described in section \ref{sec:algo:align:get_CCC}.
\end{enumerate}

Each particle $p$ has its own CCC score. As you'll see in the next section, if we're not able to find a higher CCC score during the angular search, the particle will keep its initial rotation ${[\bm{R}_{init}]}_{p}$ and translation ${[\bm{T}_{init}]}_{p}$.

\subsubsection{Angular search} \label{sec:algo:align:angular_search}

In tomography, the particles are represented as 3D volumes (i.e. subtomograms). As such, we can either rotate the reference into the microscope frame \textit{or} rotate the particle into the reference frame. The later is not possible in SPA and allows us to apply the symmetry of the particle before comparing it with the (symmetrized) reference. The difficulty here is that, the center of symmetry, or in the case of central symmetry, the position of the symmetry axis, is not necessarily correct and any error will greatly damage the particle if the symmetry is applied. To limit this effect, we first rotate the reference into the microscope frame to get a first estimate of the translation and have a better estimate of the position of the symmetry axis. Then, using this translation, we align the particle to reference to get the CCC score.

The in- and out-of-plane angles $[\Theta_{out},\ \Delta_{out}, \Theta_{in},\ \Delta_{in}]$ registered in \code{Raw\_angleSearch} are converted into a set of $r \times 3$ Euler angles ($\phi_{r},\ \theta_{r},\ \psi_{r}$). These rotations are finally converted into $r$ rotation matrices $\bm{R}_{r}$. For each particle $p$ and for each rotation $r$:

\begin{enumerate}
    \item \textbf{Get the current rotation}: The rotation of the particle is defined as ${[\bm{R}_{search}]}_{p,r} = {[\bm{R}_{init}]}_p \bm{R}_{r}$. Remember, ${[\bm{R}_{init}]}_p = \bm{R}_{n\text{-}1,p}$.
    
    \item \textbf{Get the current translation}: The current translation of the particle didn't change since the last cycle, it is still equal to  ${[\bm{T}_{search}]}_{p,r} = {[\bm{T}_{init}]}_{p} = {[\bm{T}_{orig}]}_p$.
    
    \item \textbf{Update the translation}: Using the ${[\bm{R}_{search}]}_{p,r}$ and ${[\bm{T}_{search}]}_{p,r}$ defined above, we calculate ${[\bm{T}_{S\text{-}s}]}_{p,r}$, as described in section \ref{sec:algo:align:get_translation}. The new translation is defined as ${[\bm{T}_{search}]}_{p,r} = {[\bm{T}_{init}]}_{p,r} + {[\bm{T}_{S\text{-}s}]}_{p,r}$.
    
    \item \textbf{Get the CCC score}: Using the ${[\bm{R}_{search}]}_{p,r}$ and ${[\bm{T}_{search}]}_{p,r}$ defined above, we calculate the CCC score of this transformation, ${[\bm{\mathrm{CCC}}_{search}]}_{p,r}$, as described in section \ref{sec:algo:align:get_CCC}.
    
\end{enumerate}

We now have, for each particle $\bm{s}_{p}$, the initial score ${[\bm{\mathrm{CCC}}_{init}]}_{p}$ and a set of ${[\bm{\mathrm{CCC}}_{search}]}_{p,r}$ score. The maximal ${[\bm{\mathrm{CCC}}_{search}]}_{p,r}$ score is defined as ${[\bm{\mathrm{CCC}}_{search}]}_{p} = \max_{r}({[\bm{\mathrm{CCC}}_{search}]}_{p,r})$ and the corresponding rotation and translation are referred to as ${[\bm{R}_{search}]}_p$ and ${[\bm{T}_{search}]}_p$.

The current best score is defined as follow:
\begin{equation}
    {[\bm{\mathrm{CCC}}_{best}]}_{p} = \max \left( {[\bm{\mathrm{CCC}}_{init}]}_{p}, {[\bm{\mathrm{CCC}}_{search}]}_{p} \right)
\end{equation}
As such, if ${[\bm{\mathrm{CCC}}_{init}]}_{p} < {[\bm{\mathrm{CCC}}_{search}]}_{p}$, the particle is assigned to a new transformation, defined by the rotation ${[\bm{R}_{best}]}_p = {[\bm{R}_{search}]}_p$ and the translation ${[\bm{T}_{best}]}_p = {[\bm{T}_{search}]}_p$. Otherwise, the particle keeps its original transformation, ${[\bm{R}_{best}]}_p = {[\bm{R}_{init}]}_p$ and ${[\bm{T}_{best}]}_p = {[\bm{T}_{init}]}_p$.

\subsubsection{Angular search refinement}

If the angular search goes out of plane, it is refined, using a finer sampling, around the current best rotation and best translation. To do so, a new set of $r' \times 3$ Euler angles is calculated and transformed into $r'$ rotation matrices $\bm{R}_{p,r'}$. Then, for each particle $p$ and for each rotation $r'$:

\begin{enumerate}
    \item \textbf{Get the current rotation}: The rotation of the particle is defined as ${[\bm{R}_{search'}]}_{p,r'} = {[\bm{R}_{best}]}_p \bm{R}_{r'}$.
    % From what I remember, we define new rotations and apply it to the original rotation, so it should be R_init * R_r', but I think it is more meaningful to say that we "look around" the current best rotation (=refinement). In any case, it is equivalent to what we do in practice...

    \item \textbf{Update the translation}: Using the ${[\bm{R}_{search'}]}_{p,r'}$ and ${[\bm{T}_{best}]}_{p}$ defined above, we calculate ${[\bm{T}_{S\text{-}s}]}_{p,r'}$, as described in section \ref{sec:algo:align:get_translation}. The new translation is updated, such as ${[\bm{T}_{search'}]}_{p,r'} = {[\bm{T}_{best}]}_{p} + {[\bm{T}_{S\text{-}s}]}_{p,r'}$.
    
    \item \textbf{Get the CCC score}: Using the ${[\bm{R}_{search'}]}_{p,r'}$ and ${[\bm{T}_{search'}]}_{p,r'}$ defined above, we calculate the CCC score of this transformation, ${[\bm{\mathrm{CCC}}_{search'}]}_{p,r'}$, as described in section \ref{sec:algo:align:get_CCC}.
\end{enumerate}

Similarly to the first angular search, we now have for each particle $\bm{s}_p$, the current best score ${[\bm{\mathrm{CCC}}_{best}]}_{p}$ calculated at the previous section and a set of ${[\bm{\mathrm{CCC}}_{search'}]}_{p,r'}$. The highest ${[\bm{\mathrm{CCC}}_{search'}]}_{p,r'}$ score is defined as ${[\bm{\mathrm{CCC}}_{search'}]}_{p} = \max_{r'}({[\bm{\mathrm{CCC}}_{search'}]}_{p,r'})$ and the corresponding rotation and translation are referred as ${[\bm{R}_{search'}]}_p$ and ${[\bm{T}_{search'}]}_p$.

The current best score is defined as follow:
\begin{equation}
    {[\bm{\mathrm{CCC}}_{best'}]}_{p} = \max \left( {[\bm{\mathrm{CCC}}_{best}]}_{p}, {[\bm{\mathrm{CCC}}_{search'}]}_{p} \right)
\end{equation}
As such, if ${[\bm{\mathrm{CCC}}_{best}]}_{p} < {[\bm{\mathrm{CCC}}_{search'}]}_{p}$, the current best estimate is updated and the particle is assigned to a new transformation, defined by the rotation ${[\bm{R}_{best'}]}_p = {[\bm{R}_{search'}]}_p$ and the translation ${[\bm{T}_{best'}]}_p = {[\bm{T}_{search'}]}_p$. Otherwise, the particle keeps its transformation, ${[\bm{R}_{best'}]}_p = {[\bm{R}_{best}]}_p$ and ${[\bm{T}_{best'}]}_p = {[\bm{T}_{best}]}_p$.

\subsubsection{Final translation and update}

Finally, a final translation ${[\bm{T}_{S\text{-}s}]}_p$ will be calculated as described in section \ref{sec:algo:align:get_translation}, using the final rotation ${[\bm{R}_{final}]}_p = {[\bm{R}_{best}]}_p$ (or if there was a refinement, ${[\bm{R}_{final}]}_p = {[\bm{R}_{best'}]}_p$) and current best translation ${[\bm{T}_{best}]}_p$ (or ${[\bm{T}_{best'}]}_p$). The final translation is defined as ${[\bm{T}_{final}]}_p = {[\bm{T}_{best}]}_p + {[\bm{T}_{S\text{-}s}]}_p$ (or ${[\bm{T}_{final}]}_p = {[\bm{T}_{best'}]}_p + {[\bm{T}_{S\text{-}s}]}_p$)

Once the alignment is over, the particles are updated with their new rotation and translation such as:
\begin{itemize}
    \item \textbf{New rotation}: The new rotation is simply defined as $\bm{R}_{n,p} = {[\bm{R}_{final}]}_p$
    \item \textbf{New translation}: ${[\bm{T}_{final}]}_p$ contains the original translation ${[\bm{T}_{orig}]}_p$ resulting from the extraction. As the original $x,y,z$ coordinates of the particle $\bm{T}_{n\text{-}1,p}$ implicitly contains ${[\bm{T}_{orig}]}_p$, we update the coordinates such as $\bm{T}_{n,p} = \bm{T}_{n\text{-}1,p} + {[\bm{T}_{final}]}_p - {[\bm{T}_{orig}]}_p$.
\end{itemize}







%% WORKFLOW

%%%%%%%%%%%%%%%%%%
%%%%%%%%%%%%%%%%%% PREPARE MASKED REFERENCES
%%%%%%%%%%%%%%%%%%

% Get parameters and metadata of the current state of alignment.
% Define box size like in avg.
% assign a given sub-region to a GPU. Each GPU as a queue.

% load the references.

% Calculate the COM:
%   - create comMask, which centers on the region to align (Ali_mRadius).
%   - mask the reconstruction with comMask and calculate the COM with EMC_maskReference{fsc=true}.
%   - shift the reference to have the COM at the center of the reference.
%   - Save in refIMG

% The SF3D is normalized, padded to sizeCalc. Save in refWgtROT. ifftshift and save in refWGT.

% Combine low-resolution information:
%   - pad the 2 refs to 512x512, double precision
%   - sharedInfo = sum of the two padded fft (weight 0.5 for both), but only up to current resolution cutoff (maxGoldStandard).
%   - then add specific info of the reconstruction after the cutoff: sharedInfo + fft after cutoff. NOTE: of course, this is done for both half-maps.
%   - switch back to real space and reverse the padding.
%   - Update refIMG

% eraseMask <=> if Peak_mRadius
%   - calculate a rectangular mask, where 1 is within ceil(2*eraseMaskRadius) and 0 is from there up to edges (sizeCalc).

% volMask
%   - Ali_mRadius mask.

% Bandpass filter: used to remove information after the FSC cutoff (use the spherical FSC only).
%   - Extract the FSC information from masterTM.cycleNumber.fitFSC.Raw1;
%   - bandpassFilt{1] = BH_multi_cRef(fscInfo, radialGrid)
%       - extra the shellFSC, replace neg value with 0, and fit polynomial curve. This is on the spherical FSC only.
%       - extract the forceMask (gaussian fall-off at the cutoff).
%       - cRef = fit(osX, sqrt( abs( 2 * fitFSC(osX) / (1+fitFSC(osX))) ) .* adHocMTF .* forceMask, 'cubicSpline');
%       - bandpassFilt{1] = cRefFilter + iConeMask .* reshape(cRef(radialGrid),size(radialGrid));  
%   - wdgBinary = bandpassFilt{1} > 0.01

% Save final volumes (for each half-set)
%   - ref_FT1 = Apply volMask to refIMG, and normalize within the alignment mask. Set mean and std. Finally take the conjugate! 
%   - ref_FT2 = same but with peakMask, which depends on the particleRadius. Here it is not the conjugate.

%%%%%%%%%%%%%%%%%%
%%%%%%%%%%%%%%%%%% ANGULAR SEARCH
%%%%%%%%%%%%%%%%%%

% Calculate all of the angles [phi, theta, psi-phi].

%%%%%%%%%%%%%%%%%%
%%%%%%%%%%%%%%%%%% SAMPLING FUNCTION
%%%%%%%%%%%%%%%%%%

% Check that the sampling functions exists for every tilt-series at this binning, otherwise, calculate them.

%%%%%%%%%%%%%%%%%%
%%%%%%%%%%%%%%%%%% ALIGNMENT
%%%%%%%%%%%%%%%%%%

% For each process (GPU), start the alignment only if nothing found in alignResume.
% Load the sampling functions for that sub-region (actually the same for the entire stack).
%   - apply sort of radial weighting depending on the sampling (1/sqrt(bin)).
% Get ready to extract the particle by looking at the dimensions and get list of particles.

% For every valid particle:
%   - 1) get the sampling function of that particle.

%   - 2) get coordinates and angles, and extract it similarly as in avg (BH_isWindowValid).
%       - If the particle is fully within the sub-region, then it's fine, otherwise mark it as not valid.
%       - Load this subtomogram into device memory and pad if necessary to take into account missing voxels in alignment mask.

%   - 3) Get the initial CCC score:
%       - rotate the particle with its current rotation (within alignment mask), apply symmetry if any.
%       - rotate its SF3D with the same rotation, within bandpassFilt.
%       - apply the mask to the rotated particle, mask it with alignment mask, bandpass filter with bandpassFilt and normalize within alignment mask.
%       - calculate the initial CCC score with BH_multi_xcf_Rotational (see below) and store it in cccInitial = {iref (1), particleIDX, 0, 0, 0, iCCC, 1, 0}.

%   - 3) For every theta, every phi (azimuth) and every psi:
%       - translation estimate:
%           - a) rotate the reference in the particle frame, as well as its SF3D. The rotation is only within the particleMask and bandpassFilt, respectively.
%                Apply the remaining shifts from the particle coordinates (during extraction you round to be in pixel space).
%           - b) Apply the particleMask to the rotated ref_FT2 again and normalize within the mask.
%           - c) Apply the particleMask to the particle, bandpass filter with bandpassFilt.
%           - d) extract the translation with BH_multi_xcf_Translational:
%               - i) apply the total SF3D to the (unrotated) particle.
%               - ii) the particle's SF3D to the conjugate of the rotated reference ref_FT2.
%               - iii) multiply both of them in Fourier space, ifft and fftshift to have the "centered" CCCmap.
%               - iv) for the COM, set the min of CCCmap to 0.
%               - v) extract the x, y and z coordinates of the strongest peak.
%               - vi) calculate the COM around this peak (7x7x7 box) and add it to the peak coordinate.
%               - vii) Round this updated peak coordinate and repeat the step vi).
%           - e) store in cccStorageTrans: {iref (1), particleIDX, phi, theta, psi-phi, 0, 0, coord + translation}.
%       - CCC score:
%           - a) using the first estimate of the translation in cccStorageTrans, shift and rotation the particle. Apply any symmetry.
%           - b) rotate the particle's SF3D within bandpassFilt.
%           - c) apply the mask to the rotated particle, mask it with alignment mask, bandpass filter with bandpassFilt and normalize within alignment mask.
%           - d) calculate the CCC score with BH_multi_xcf_Rotational:
%               - i) numerator = real( rotPart * wdgRef * ref_FT1 * wgdMask ). NOTE: ref_FT1 is already the conjugate.
%               - ii) denominator = sum(abs(rotPart * wdgRef)^2) * sum(abs(REF_FT1 * wdgMask)^2)
%               - iii) CCC = sum(numerator) / sqrt(denominator)
%           - e) store in cccStorage2 (just an update of cccStorageTrans): {iref (1), particleIDX, phi, theta, psi-phi, CCC, 1, translation}.

%   4) If the initial CCC score in cccInitial is greater than the highest newly calculated CCC score in cccStorage2, don't update and use old rotation and no shift.
%      The sorted, and updated with cccInitial, cccStorage2 is saved in cccPreRefineSort.

%   5) If refine (out of plane search):
%       - Extract the angles from cccPreRefineSort and set finer increments: theta = psi = 2 times smaller, phi = 4 times smaller.
%       - If theta, i.e. if in plane, reduce the psi increment even more: psiInc = floor(sqrt(thetaInc))
%       - Set new angular search (refine around the best angles).
%       - for each phi, theta and psi: 
%           - Same as first angular search but for the translation estimate use the translation estimate from the best rotation.

%   6) Get the final translation using the current best rotation and best translation estimate.
%       - Get the rotation and translation and apply it to ref_FT2. Apply the rotation to its SF3D.
%       - Mask the reference with particle mask and normalize within the mask.
%       - Mask the unrotated particle with particle mask, bandpassFilt, and normalize with mask.
%       - calculate the peakCoord with BH_multi_xcf_Translational and update the translation estimate with it.
%       - Subtract shitVAL (shift used for extraction) to the final translation.

%   7) Print final result: CCC, rotation, shift; before and after alignment (+refinement)

% Once the alignment is done for every particle of a given sub-region, save the results (cccStorageBest) in alignResume.

% Save the new rotations and coordinates in the metadata and update also class for particles that are not valid.





\newpage

\subsection{Tilt-series refinement} \label{sec:algo:tomoCPR}


Subtomogram averaging provides accurate estimates of both particle positions and high SNR reconstructions. It is thus possible to leverage this information for improving the alignment of a tilt-series.


\subsubsection{Reconstruct the synthetic tomogram}

The first is to reconstruct the full tomogram $\bm{V}$ in the same way we reconstructed the sub-regions tomograms in section \ref{sec:algo:ctf_3d}. Similarly to the sub-region tomograms used for subtomogram averaging and alignment, this tomogram is CTF-phase corrected. To save some precious run time and because this tomogram is only used to refine the fiducial positions, {\emClarity} does not follow the so-called ``3D CTF-correction``. Indeed, the correction is forced to have only one $z$ slab (section \ref{sec:algo:ctf_3d:defocus_step}). In other words, the thickness of the specimen is not taken into account during the correction, but only the tilts. Moreover, the center of mass in $z$ is not adjusted and the spatial model is not defined (section \ref{sec:algo:ctf_3d:spatial_model}).

% ASK BEN: The tomogram is standardized ($\sigma=1$) and weighted based on the particle mass:
%          divide the tomogram by rmsScale*rms(tomogram); rmsScale = sqrt(particleMass).

Once that the full tomogram is reconstructed, the subtomograms are replaced by their corresponding half-maps. For each particle $i$:

\begin{enumerate}
    \item \textbf{Get the coordinates of the particle}: The $x,\ y,\ z$ coordinates of the particle, that are saved in the metadata, corresponds to the 3D coordinates within the sub-region tomogram, with the origin at the lower left corner. In this section, we are working with the full tomogram $\bm{V}$, i.e. the entire field of view, so the coordinates must be adjusted to $\bm{V}$.

    \item \textbf{Get the reference in the microscope frame}: The particle is attached to a rotation $\bm{R}_i$. Moreover, as explained in section \ref{sec:algo:avg:subtomo_avg} step \textbf{1.b}, the particles $p$ are attached to a translation ${[\bm{T}_{orig}]}_p$. The half-map $\bm{S}$ is rotated by $\bm{R}^{T}_i$ and translated by ${[-\bm{T}_{orig}]}_i$ (note \ref{note:ref2mic_frame}). The same transformation is applied to the soft-edged molecular mask $\bm{M}_{mol}$ (section \ref{sec:algo:avg:molecular_mask}). We will refer to the transformed reference and transformed mask as $\bm{S}_p$ and ${[\bm{M}_{mol}]}_p$.
    % I'm skipping the fact that we actually compute the fsc mask on ref1+ref2 and applying a spherical mask.
    
    \item \textbf{Replace the particle's density by the reference's density}: The voxels of the full tomogram $\bm{V}$ corresponding to the particle $\bm{s}_p$ are replaced by the masked reference, such as:
    \begin{equation}
        \bm{V}(\bm{s}_p) = \left(1-{[\bm{M}_{mol}]}_p \right)\bm{s}_p + \bm{S}_p {[\bm{M}_{mol}]}_p
    \end{equation}
    where $\bm{V}(\bm{s}_p)$ refers to the voxels of $\bm{V}$ that corresponds to the particle $\bm{s}_p$.
\end{enumerate}


%   - 1) Reconstruct the full tomogram:
%       - a) Get the thickness of the reconstruction t (maxZ):
%           - For each sub-regions, get the Zmax-Zmin + Zshift, and save maxZ as the thickest.
%       - b) Call ctf 3d, but reconstruct the entire field of view, use the maxZ defined above, one Z section, no surfaceFit.
%            Therefore, this is not a 3D correction: the tilt is corrected, but no thickness (one Z section...).
%            You don't use the particle positions here: no offsets.

%   - 2) Standardize the tomogram and weight it based on the particle mass:
%       - a) divide the tomogram by rmsScale*rms(tomogram); rmsScale = sqrt(particleMass).

% - 3) For each sub-region:
%       - a) Get the exposure (first view is exposure=0).
%       - b) Get the defocus values z: _align.defocus
%       - c) Get the xf file (no rotation, no shift): _align.XF
%       - d) Get the sub-regions coordinates (relative to the full tomo).
%       - e) Get the particleMask = boxsize of the reference, sphere of particleRadius * fsc reference mask of ref1+ref2

%       - f) For each subtomo: Reprojection of the subtomograms.
%           - i) Get rotation and position of particle: prjVector (relative to the full tomogram, center).
%                There is a preShift = [-0.5,-0.5,0.5] applied. This is for IMOD I think, pixel vs coord space.
%           - ii) Extract the subtomogram pixel coordinates indVal and shiftVal, using its subregions tomogram boundaries.
%                 This is like avg or alignRaw.
%           - iii) Put the reference of the particle (fsc 1 or 2) in the mic frame: rot' and then shiftVal.
%                  Apply the same transformation to the ref mask and apply it to the resampled ref.
%           - iv) Replace the full tomogram voxels by the reference: tomo * (1-refMask) + ref
%                 Multiplying by 1-refMask to take into account the taper.
%           - v) Save the 3d position of the particle. The positions are 90deg rotx, lower left corner full tomo.
%               _ali*.coord: x,y,z,fid_id
%           - vi) Save the 2d position of the particle for each image in the stack:
%               _ali*.defAng = fid_id,section,def; def = z_tilt + z_subtomo (coord(x,y,z) * tilt), in binned pixel.
%               _ali*.coord_start = fid_id,subregion,particle_id,def,def_shift,def_angle(deg),rotm,pre_exp,post_exp,fsc_group.
%                                   The def here is not as defAng; here it is not tilted, just prjVector + z_tilt.

\subsubsection{Reproject the synthetic tilt-series} \label{sec:algo:tomoCPR:reproject_coords}

We want to reproject the ``synthetic'' tilted views \emph{and} the particles coordinates. Both reprojections are calculated by {\tilt}:
\begin{itemize}
    \item \textbf{Reproject the coordinates}: For the tilt-series alignment, {\tiltalign} needs the $x,\ y$ coordinates of the particles, for each view of the tilt-series. Fortunately, {\tilt} can reproject the $x,\ y,\ z$ coordinates of the particles and if the defocus value are know, for each view, it can also calculate the defocus of each particle, for each view, while accounting for the local alignments that were used during the reconstruction.
    % As such, we run {\tilt} with the following entries:
    % table
    
    \item \textbf{Reproject the tilt-series}: The synthetic tomogram, calculated at the previous section, is reprojected into an aligned synthetic tilt-series using {\tilt}. Of course, it takes into account the eventual local alignments that were used to reconstruct the tomogram.
\end{itemize}

\begin{note}{\tilt} and {\tiltalign} operates with the $y$ axis, i.e. the \code{SLICE}, in the third dimension. So both the ``synthetic'' tomogram and the $x,\ y,\ z$ coordinates are rotated by 90\textdegree\ around $x$.
\end{note}


%  - v) Save the 3d position of the particle. The positions are 90deg rotx, lower left corner full tomo.
%               _ali*.coord: x,y,z,fid_id
%   - 4) Reprojection 3d from 2d:
%       - a) Save _ali*.coord into model file: .3dfid and reproject (at the tilt angles) the 3dfid coordinates with tilt = coordPrj.
%            Each fiducial has its defocus value reprojected as well, as in .defAng, but with local alignments compensated = defAngTilt.
%       - b) Rotate -rx and save the synthetic tomogram .tmpRot.
%       - c) Reproject with tilt at the same tilt angles than original tilts:
%           - The reprojection is divided into chunks, in Y.
%           - COSINTERP 0, THICNESS maxZ, local file if any.

\subsubsection{Refine the fiducial positions}

The synthetic tomogram is now reprojected into a synthetic tilt-series. Tiles around each projected high SNR subtomogram origin are masked out, convolved with the CTF of the raw data projection at that point and aligned to the raw data.

By default, {\emClarity} is setting the maximal number of particle which are going to be used as fiducial to 1800. This value can be changed with the \code{tomoCPR\_randomSubset} entry. If there is more particles than the allowed number of fiducial, a random subset of particles will be selected.

We will refer to the raw aligned tilt-series as $\bm{I}_{raw}$ and synthetic tilt-series as $\bm{I}_{synt}$. Before starting the refinement, each view ${[\bm{I}_{raw}]}_i$ and ${[\bm{I}_{synt}]}_i$ are centered and standardized, first globally, then locally. Combined with the sampling mask, calculated in section \ref{sec:algo:defocus_estimate:transform}, it allows us to define a mask ${[\bm{M}_{eval}]}_i$ excluding regions that are not sampled or that significantly varies from the rest of the data, like carbon, contaminants, etc.

Before refining the fiducial positions, a global shift estimate is calculated, for each view $i$.
\begin{enumerate}
    \item \textbf{Calculate the cross-correlation map}: The cross-correlation between ${[\bm{I}_{raw}]}_i$, which is not CTF corrected, and ${[\bm{I}_{synt}]}_i$, which is multiplied by the CTF, is defined as follow:
    \begin{equation}
        \bm{\mathrm{CC}}_{i} = \mathcal{F}^{-1} \left\{ \bm{W}_{low\text{-}pass}\ {[\bm{W}_{ctf}]}_i\ \mathcal{F}\left\{ {[\bm{I}_{raw}]}_i \right\}\ \overline{\mathcal{F}\left\{ {[\bm{I}_{synt}]}_i \right\}}\ \left|{[\bm{W}_{ctf'}]}_i\right|\ \right\} \bm{M}_{peak\text{-}global}
    \end{equation}
    where $\bm{W}_{low\text{-}pass}$ is a low-pass filter, with a low-cutoff set by \code{tomoCprLowPass}. ${[\bm{W}_{ctf}]}_i$ is the astigmatic 2D CTF of the view $i$, with envelope. $|{[\bm{W}_{ctf'}]}_i|$ is the astigmatic 2D CTF without envelope and is used to modulate the amplitudes of the reprojected reference to better match the raw data. $\bm{M}_{peak\text{-}global}$ is a spherical mask and limits the translation to $\sim 10$ to $20$\si{\angstrom}.
    
    \item \textbf{Get highest peak position}: This step is identical to the last step of section \ref{sec:algo:align:get_translation} and outputs a translation ${[\bm{T}_{global}]}_i$.
    
    \item \textbf{Apply the global translation estimate}: ${[\bm{I}_{raw}]}_i$ is translated by ${[\bm{T}_{global}]}_i$, using linear interpolation.
\end{enumerate}

%       - c) Calculate the ctf of the projection, without envelope. PhaseOnly=-0.15 and not 1? HqzUnMod is the CTF before cutoffs and without PhaseOnly.
%       - d) Calculate the CCmap = fftshift(ifftn( bandPassPrj * fftn(dataPrj) * abs(HqzUnMod) * conj(fftn(refPrj)*Hqz) ))
%            To match the amplitudes, given that the projected reference is amplitude corrected, I multiplying fftn(refPrj) with abs(ctf). Why not do same for local refinement?
%       - e) Get the peak estPeak = subtract with min, apply globalPeakMask, take the max, get 7x7 box around it, calculate the COM and add to max. This is like usual.
%       - f) Shift the dataPrj by estPeak.

Then, the $x$ and $y$ position of each particle, for each view $i$ is refined as follow:
\begin{enumerate}
    \item \textbf{Extract the particle tile}: For both $\bm{I}_{raw}$ and $\bm{I}_{synt}$, a tile of $1.5 \times \code{particleRadius}$ is extracted at the particle reprojected coordinate, centered and standardized. If the tiles overlap with the evaluation mask ${[\bm{M}_{eval}]}_i$, the particle is ignored and will not be used as fiducial.
    
    \item \textbf{Calculate the translation between the raw and the synthetic tile}: The tiles are padded 2 times in real space, masked and the CC map is calculated as follow:
    \begin{equation}
        \bm{\mathrm{CC}}_{p,i} = \mathcal{F}^{-1} \left\{ \bm{W}_{low\text{-}pass}\ {[\bm{W}_{ctf}]}_{p,i}\ \mathcal{F}\left\{ {[\bm{I}_{raw}]}_{p,i} \right\}\ \overline{\mathcal{F}\left\{ {[\bm{I}_{synt}]}_{p,i} \right\}}\ \right\} \bm{M}_{peak\text{-}local}
    \end{equation}
    where ${[\bm{W}_{ctf}]}_{p,i}$ is the anisotropic 2D CTF at the particle position, defined by the defocus value $\bm{\mathrm{z}}_{p,i}$. $\bm{M}_{peak\text{-}local}$ is a mask controlled by \code{Peak\_mType} and \code{Peak\_mRadius} and is used to restrict the translation. By default, it is set to $0.4 \times \code{particleRadius}$.
    
    \item \textbf{Get the translation estimate}: This step is identical to the last step of section \ref{sec:algo:align:get_translation} and outputs a translation ${[\bm{T}_{local}]}_{p,i}$.
\end{enumerate}

At this end of this procedure, each fiducial is translated by ${[\bm{T}_{orig}]}_{p,i} + {[\bm{T}_{global}]}_i + {[\bm{T}_{local}]}_{p,i}$ and saved for the next step.

The defocus value can be refined by sampling a range of defocus value around the current estimate in order to maximize $\mathrm{CC}_{p,i}$. The defocus search is set by $\bm{\mathrm{z}}_{p,i} \pm $ \code{tomoCprDefocusRange}, with a step of \code{tomoCprDefocusStep}. For each image $i$, the defocus shifts $\bm{\Delta \mathrm{z}}_{p,i}$ that maximize $\mathrm{CC}_{p,i}$ are averaged and this average will be added to the current the defocus value of the image.

\subsubsection{Align and transform the tilt-series}

The fiducials are now aligned, so we can use them to find a new geometric model with {\tiltalign}. A bash script is saved in \code{mapBack<n>} and contains the parameters that are used for the alignment. Most of them are accessible directly via the {\emClarity} parameter file (table \ref{param:tomoCPR}).

Finally, the raw unaligned tilt-series in \code{fixedStacks} is transformed using this new geometric model. This last part is done via \code{ctf update} and is similar to section \ref{sec:algo:defocus_estimate} to the exception that this new alignment is relative to the aligned tilt-series. As such, the new rotation and shifts are added to the original ones, and the tilt angles are updated. This new transformation is saved in a new table \ref{tab:ctf_tlt}. The beads coordinates are updated as well and erased.

%% WORKFLOW

% Get the reference:
%   - un-mount the reference

%%% For every tilt-series saved in mapBackGeometry:

%   - 1) Reconstruct the full tomogram:
%       - a) Get the thickness of the reconstruction t (maxZ):
%           - For each sub-regions, get the Zmax-Zmin + Zshift, and save maxZ as the thickest.
%       - b) Call ctf 3d, but reconstruct the entire field of view, use the maxZ defined above, one Z section, no surfaceFit.
%            Therefore, this is not a 3D correction: the tilt is corrected, but no thickness (one Z section...).
%            You don't use the particle positions here: no offsets.

%   - 2) Standardize the tomogram and weight it based on the particle mass:
%       - a) divide the tomogram by rmsScale*rms(tomogram); rmsScale = sqrt(particleMass).

%   - 3) For each sub-region:
%       - a) Get the exposure (first view is exposure=0).
%       - b) Get the defocus values z: _align.defocus
%       - c) Get the xf file (no rotation, no shift): _align.XF
%       - d) Get the sub-regions coordinates (relative to the full tomo).
%       - e) Get the particleMask = boxsize of the reference, sphere of particleRadius * fsc reference mask of ref1+ref2

%       - f) For each subtomo: Reprojection of the subtomograms.
%           - i) Get rotation and position of particle: prjVector (relative to the full tomogram, center).
%                There is a preShift = [-0.5,-0.5,0.5] applied. This is for IMOD I think, pixel vs coord space.
%           - ii) Extract the subtomogram pixel coordinates indVal and shiftVal, using its subregions tomogram boundaries.
%                 This is like avg or alignRaw.
%           - iii) Put the reference of the particle (fsc 1 or 2) in the mic frame: rot' and then shiftVal.
%                  Apply the same transformation to the ref mask and apply it to the resampled ref.
%           - iv) Replace the full tomogram voxels by the reference: tomo * (1-refMask) + ref
%                 Multiplying by 1-refMask to take into account the taper.
%           - v) Save the 3d position of the particle. The positions are 90deg rotx, lower left corner full tomo.
%               _ali*.coord: x,y,z,fid_id
%           - vi) Save the 2d position of the particle for each image in the stack:
%               _ali*.defAng = fid_id,section,def; def = z_tilt + z_subtomo (coord(x,y,z) * tilt), in binned pixel.
%               _ali*.coord_start = fid_id,subregion,particle_id,def,def_shift,def_angle(deg),rotm,pre_exp,post_exp,fsc_group.
%                                   The def here is not as defAng; here it is not tilted, just prjVector + z_tilt.

%   - 4) Reprojection 3d from 2d:
%       - a) Save _ali*.coord into model file: .3dfid and reproject (at the tilt angles) the 3dfid coordinates with tilt = coordPrj.
%            Each fiducial has its defocus value reprojected as well, as in .defAng, but with local alignments compensated = defAngTilt.
%       - b) Rotate -rx and save the synthetic tomogram .tmpRot.
%       - c) Reproject with tilt at the same tilt angles than original tilts:
%           - The reprojection is divided into chunks, in Y.
%           - COSINTERP 0, THICNESS maxZ, local file if any.

%   - 5) Prepare the alignment:
%       - CTFSIZE = 2 * 1.5 * particleRadius, up to best Fourier = box size
%       - ctfMask = spherical mask that covers the entire tile -padding.
%       - peakMask = spherical 0.4 * particleRadius. If eraseMask = define own type and radius with Peak_mType and Peak_mRadius.
%       - bandPassPrj = size of the tilt-series. lowpassCutoff = tomoCprLowPass, between 10 and 24A (and Nyquist).
%       - globalPeakMask = closest even int to max(2, ceil(10/pixSize)); look around +/- 10-20A. The mask is at the center of the CCmap of the stack.

%   - 6) Fiducial: select the fiducial to follow. Take all if < tomoCPR_randomSubset, otherwise take a random subset.

%   - 7) For each view:
%       - a) Load both the stack (dataPrj), synthetic stack (refPrj) and samplingMask. Resample samplingMask to current binning.
%       - b) Get dataPrj ready:
%           - i) center and standardize globally = dataPrj
%           - ii) then calculate the local mean, subtract it = dataRMS. The local window is 256x256 unbinned,
%                 or 64x64 unbinned for the local rms (for the mask, not for the local scaling).
%           - iii) Remove outliers: calculate on the dataRMS the local rms, the global mean and rms.
%                  evalMask = dataRMS > (mRms - 2*sRms) & ~samplingMask
%                  This remove from the evalMask the regions that are not sampled, and carbone, etc.
%           - iv) Local scaling: subtract the local mean to dataPrj, and divide by local rms of this centered dataPrj.
%                OR Option: BH_whitenNoiseSpectrum
%           - v) Global scaling: center and standardize dataPrj AND refPrj with their mean and rms.

%       - c) Calculate the ctf of the projection, without envelope. PhaseOnly=-0.15 and not 1? HqzUnMod is the CTF before cutoffs and without PhaseOnly.
%       - d) Calculate the CCmap = fftshift(ifftn( bandPassPrj * fftn(dataPrj) * abs(HqzUnMod) * conj(fftn(refPrj)*Hqz) ))
%       - e) Get the peak estPeak = subtract with min, apply globalPeakMask, take the max, get 7x7 box around it, calculate the COM and add to max. This is like usual.
%       - f) Shift the dataPrj by estPeak.

%       - g) For each fiducial in this projection:
%           - IF ctfCalc:
%               - if sqrt(2)*pixelSize > min_res_for_ctf_fitting(10), then, turn-off the defocus refinement.
%              
%           - take the X and Y coordinates of the particles from .coordPrj
%           - tileRadius = 1.5 * PARTICLE_RADIUS. oxEval/oyEval = X/Y +/- PARTICLE_RADIUS.
%           - If any 0 in the evalMask, within this the particle radius OR if the tile is out-of-bounds, the particle is ignored.
%           - extract the tile, from dataPrj and refPrj. center and standardize.
%           - Pad the images to CTFSIZE (oversample) and apply ctfMask to the tiles.
%           - calculate fft of tiles, lowpass to lowPassCutoff, highpass to 400. Take the conj for the refPrj and multiply by the CTF of particle (using defAngTilt).
%               - if ctfCalc:
%                   - highpass to 40, lowpass to min(sqrt(2)*pixelSize, min_res_for_ctf_fitting).
%                   - try a range of defocus, set by tomoCprDefocusRange and tomoCprDefocusStep.
%           - dXY = calculate the CCmap, apply peakMask and get the peak coordinates as usual.
%                   Take into account the x/y shift due to extraction of the tiles AND estPeak.
%           - if calcCTF:
%               - take the defocus that gave the highest peak. Take the mean of these defoci for each fiducial in this image: this is the defocus shift to apply to the defocus at the titl-axis of this given projection. This is saved in _ctf.defShifts
%           - Save the new coordinates in .coordFIT: particle_id, fid_id, dXY

%   - 8) Call tiltalign - refine the tilt-series alignment:
%       - RotOption	1: for each view having an independent rotation
%       - TiltOption 5: to automap groups of tilt angles (for linearly changing values), TiltDefaultGrouping = 5
%       - MagOption 1: to vary all magnifications independently
%  XStretchOption	0
% SkewOption	0
% BeamTiltOption	0
% XTiltOption	0
% ResidualReportCriterion	0.001
% RobustFitting
% KFactorScaling 0.458: k_factor_scaling = 10 / sqrt(nFidsTotal)
% LocalAlignments
% LocalRotOption 1
% LocalRotDefaultGrouping 3
% LocalTiltOption 5
% LocalTiltDefaultGrouping 5
% LocalMagOption 1
% LocalMagDefaultGrouping 5


%%%%%% CTF UPDATE
% Align the tilt-series with ali2_ctf.tlt, like ctf estimate (fraction inelatic, etc.).
%   - The shifts and rotation are relative to the aligned stack, so the new alignment is added to the original one. So update ali2_ctf.tlt to make it relative to the raw tilt-series (rotm and shift).
%   - Add to def value the .defShifts (if calcCTF).
%   - Update the tilt angles.
%   - Update the beads coordinates, using the new .tltxf transformation. (rotation + mag) and erase them on the new aligned stack.
%

\newpage

\subsection{Subtomogram classification}

The heterogeneity of the data-set can be analysed by comparing individual particles with the current references. This analysis can be focused to some specific features, i.e. resolutions bands, like $\alpha$-helices, small protein domains, etc. Briefly, difference maps are calculated between each particle and the references, for each resolution bands. These maps are then analysed by Principal Component Analysis (PCA), using Singular Value Decomposition (SVD). Once the difference maps are reduced in dimensionality and described by some principal axes, they can be clustered with $k$-means or other clustering algorithms. As a result, each subtomogram is assigned to a class and a subtomogram average can be generated for each class. We will detail this entire process in the following sections.

Usually, PCA is calculated on the covariance or correlation matrix of the data. These are quite popular in tomography because we can consider the missing wedge of the particles while calculating these matrices, using constrained cross-correlation. SVD allows us to perform a PCA directly on the data, but we still need to take into account that each particle has an incomplete sampling. To solve this problem, we are going to calculate difference maps.

%Multiple copies of the objects of interest can be extracted from tomograms. As for SPA, but in 3D, it is then possible to align and average these subtomograms to generate reconstructions with improved resolution. Further characterization can be done via statistical methods (classification algorithms) allowing to describe the subtomogram population based on physical differences. This is subtomogram averaging and classification. Compared to SPA, CET collects more information per object of interest \cite{mcewen_1995}, making each particle exists as a unique 3D reconstruction, allowing for a direct analysis of the 3D heterogeneity of the subtomograms.

%To correct for differences in sampling between the reference and an individual particle, the current subtomogram average is distorted by the sampling function it is being compared to. This effectively estimates what the average particle should look like at that subtomogram position, allowing to only compare meaningful differences. The dimensionality of these differences is reduced by principal component analysis, using Singular Value Decomposition (SVD). Features of a given length scale (e.g. $\alpha$-helices, small protein domains, etc.) can be focused on and considered simultaneously by band-pass filtering the reconstructions and computing the SVD for each length scale. The singular vectors (i.e. the principal axes) describing the greatest variance for each length scale are selected and used to project the data (i.e. the principal components). Then, the principal components are clustered with $k$-means or other clustering algorithms. Finally, once every subtomogram is assigned to a class, we can reconstruct the subtomogram average for each class.

\subsubsection{Combine the half-maps} \label{sec:algo:classification:combine_maps}

First, we need to prepare the reference to which the data-set is being compared to; what we are going to classify is the difference between the particles $\bm{s}$ and this reference. With the goal-standard approach, we have two half-maps, $\bm{S}_1$ and $\bm{S}_2$. For classification, these two references are aligned by spline interpolation, using the rotation $\bm{R}_{gold}$ and translation $\bm{T}_{gold}$ calculated in section \ref{sec:algo:avg:fsc}, and averaged into one unique reference, simply referred as $\bm{S}$. Of course, this alignment is not persistent, so that extracted class averages are still independent half-sets.

\begin{note}The classification can be performed at a different sampling than the half-maps, hence the two distinct parameters \code{Cls\_samplingRate} and \code{Ali\_samplingRate}. In this case, the reference $\bm{S}$ is re-scaled to the desired pixel size, using linear interpolation.
\end{note}

\subsubsection{Resolution bands} \label{sec:algo:classification:resolution_bands}

As mentioned previously, we can restrict the analysis of the physical differences between the data-set and the reference to some $r$ specific length scales, also referred as resolution bands or features. These resolution bands are controlled by the \code{pcaScaleSpace} entry and we refer to them as $\bm{L}_r$. As we will see in more detail, the PCA is calculated for each one of these length scales. Importantly, the clustering is only calculated once and considers each length scale simultaneously to provide a richer description of the feature space.

First, we can exclude from the analysis most of the background by calculating a molecular mask $\bm{M}_{mol}$ of $\bm{S}$, as described in \ref{sec:algo:avg:molecular_mask}. The \code{Cls\_mType}, \code{Cls\_mRadius} and \code{Cls\_mCenter} parameters are used to a compute soft-edged shape mask, $\bm{M}_{shape}$. In order to constrain the analysis to the desired region, this mask is applied to the molecular mask $\bm{M}_{mol}$. This mask is saved as \code{<prefix>\_pcaVolMask.mrc}.

% The reference is then masked with $\bm{M}_{mol}$ and center and standardized.
% Then, for each resolution band $\bm{L}_{r}$, the masked reference is filtered by either a band-pass filter or a low-pass filter:

Then, for each resolution band $r$,  either a band-pass filter or a low-pass filter $\bm{W}_r$ is prepared.
\begin{itemize}
    \item \textbf{Low-pass filters}: By default, low-pass filters are used and includes frequencies from 400\r{A} to $(0.9 \times \bm{L}_{r})$\r{A}. These filters do not define ``resolution bands'' per say and most frequencies will be shared in the resulting filtered references $\bm{S}_r$, but they have produced good results nonetheless.
    \begin{note}As often, frequencies before 400\r{A} are removed to center the reference and remove large densities gradients.\end{note}
    % with the reference we don't expect gradients I think, but...
    
    \item \textbf{Band-pass filters}: If \code{test\_updated\_bandpass} is true, band-pass filters are used instead and include frequencies from 400\r{A} to 100\r{A} and a resolution band at $\bm{L}_r$.
\end{itemize}

For visualization, the reference $\bm{S}$ is filtered with the filters $\bm{W}_r$, masked with $\bm{M}_{mol}$, centered and standardized. These volumes are saved in \code{test\_filt.mrc}

% The filtered references $\bm{S}_r$ are finally centered and standardized withing $\bm{M}_{mol}$, masked again with $\bm{M}_{mol}$ and saved in \code{test\_filt.mrc}.

% This is achieved by filtering the reference $\bm{S}$ with band-pass filters, resulting into $r$ references, referred as $\bm{S}_r$.

\subsubsection{Difference maps}

What we want to classify is the difference between each particle $p$ and the reference $\bm{S}$, for each length scale $r$, while accounting for incomplete sampling of the particles. To calculate a difference map $\bm{X}_{p,r}$, we do as follow:

\begin{enumerate}
    \item \textbf{Get the particle in the reference frame}: The particle $\bm{s}_p$ is extracted from the CTF phase-multiplied sub-region tomogram. Because we want to use the entire data-set for the classification, the two half-sets must be aligned in the same way we aligned two half-maps $\bm{S}_1$ and $\bm{S}_2$ in section \ref{sec:algo:classification:combine_maps}.
    As such, the first half-set is translated by ${[\bm{T}_{orig}]}_p + \bm{T}_{gold}$ and rotated by $\bm{R}_{p} \bm{R}_{gold}$, whereas the second half-set is simply translated by ${[\bm{T}_{orig}]}_p$ and rotated by $\bm{R}_p$.

    \item \textbf{Get the sampling function of the particle in the reference frame}: The same $\bm{R}_p$ rotation (or $\bm{R}_{p} \bm{R}_{gold}$) is applied to the sampling function of the particle $\bm{w}_{p}^2$ to correctly represent the original sampling of the subtomogram.
    %\begin{note}As explained in section \myref{subsubsec:SF3D}, {\emClarity} is currently not computing a sampling function per particle, but divides the field of view by 9 strips parallel to the tilt-axis.
    %\end{note}
    
    % We don't do this exactly in this order, but I think it is clearer like this.
    \item \textbf{Calculate the difference map}: The difference map $\bm{X}_{p,r}$ between the reference and the ${p^{th}}$ particle, for the $r^{th}$ length scale, is then defined as:
    \begin{equation}
        \bm{X}_{p,r} =  \mathcal{F}^{-1} \bigg\{
                                        \Big(
                                            \underbrace{\mathcal{F} \left\{ \bm{S} \right\} \bm{w}^{2}_p}_{\mu=0,\ \sigma=1}\ -\
                                            \underbrace{\mathcal{F} \left\{ \bm{s}_{p} \right\}}_{\mu=0,\ \sigma=1}
                                        \Big) \bm{W}_{r}
                                    \bigg\} \bm{M}_{mol}
    \end{equation}
    $\bm{X}_{p,r}$ is then centered and standardized.
    To save memory and time, only the voxels within $\bm{M}_{mol}$ are saved.
    \begin{note}The decomposition will be calculated on the host, but the difference maps are calculated on the device, which is much more efficient. The \code{PcaGpuPull} entry controls how many maps $\bm{X}_{p,r}$ should be held on the device at any given time.
    \end{note}
\end{enumerate}

\subsubsection{Singular Value Decomposition} \label{sec:algo:classification:SVD}

The difference maps $\bm{X}_{p,r}$ are then linearised into column vectors and stacked into a matrix $\bm{X}$, where the number of rows is the number of voxels $v$ within $\bm{M}_{mol}$, the number of columns is the number of particles $p$ and the number of pages (the third dimension) is the number of length scales $r$.

One important step when doing a PCA is to center the variables, i.e. the voxels. In our case, it means that the rows of $\bm{X}$ should be set to have a mean equal to zero. In other words, each difference map must be subtracted by the average of all the difference maps.

Once $\bm{X}$ is ready, we can then calculate the SVD, for each length scale $r$. This is a representation of what we have:

\begin{figure}[!htb]  % Stay within section
\captionsetup{labelfont=bf}
\centering

\begin{tikzpicture}[every node/.style={anchor=north east,minimum width=1.4cm,minimum height=7mm}]
\matrix (mA) [draw,matrix of math nodes, fill=white]
{
x_{1,1,r} & x_{1,2,r} & \cdots & x_{1,p,r} \\
x_{2,1,r} & x_{2,2,r} & \cdots & x_{2,p,r} \\
\vdots    &  \vdots   & \ddots & \vdots    \\
x_{v,1,r} & x_{v,2,r} & \cdots & x_{v,p,r} \\
};

\matrix (mB) [draw,matrix of math nodes, fill=white] at ($(mA.south west)+(4.6,1.3)$)
{
x_{1,1,2} & x_{1,2,2} & \cdots & x_{1,p,2} \\
x_{2,1,2} & x_{2,2,2} & \cdots & x_{2,p,2} \\
\vdots    &  \vdots   & \ddots & \vdots    \\
x_{v,1,2} & x_{v,2,2} & \cdots & x_{v,p,2} \\
};

\matrix (mC) [draw,matrix of math nodes, fill=white] at ($(mB.south west)+(4.6,1.3)$)
{
x_{1,1,1} & x_{1,2,1} & \cdots & x_{1,p,1} \\
x_{2,1,1} & x_{2,2,1} & \cdots & x_{2,p,1} \\
\vdots    &  \vdots   & \ddots & \vdots    \\
x_{v,1,1} & x_{v,2,1} & \cdots & x_{v,p,1} \\
};

\draw[dashed](mA.north east)--(mC.north east);
\draw[dashed](mA.south east)--(mC.south east);

% Basis
\draw[arrow](mC.north west)--(mA.north west) node[midway,sloped,above] {$r$ length scales};
\draw[arrow](mC.north west)--(mC.south west) node[midway,sloped,below] {$v$ voxels};
\draw[arrow](mA.north west)--(mA.north east) node[midway,sloped,above] {$p$ particles};

% fill opacity=0
% Example of difference maps
\node (rect1) at ($(mC)+(-1.5,1.42)$) [draw,minimum width=1.2cm,minimum height=2.85cm] {};
\node[below=0.5cm of rect1] (txt1) {$\bm{X}_{1,1}$};
\draw[arrow3](rect1.south)--(txt1.north);

\node (rect2) at ($(mC)+(-0.1,1.42)$) [draw,minimum width=1.2cm,minimum height=2.85cm] {};
\node[below=0.5cm of rect2] (txt2) {$\bm{X}_{2,1}$};
\draw[arrow3](rect2.south)--(txt2.north);

% draw arrow for SVD
\node[right=3.8cm of mA] (svdr) {$\bm{U}_r \bm{S}_r \bm{V}_{r}^{T}$};
\draw[arrow3](mA.east)--(svdr.west) node[below, midway] {decompose} (svdr);
\draw[dashed](svdr.east)--($(svdr.east)+(1.8,0)$);

\node[right=3.8cm of mB] (svd2) {$\bm{U}_2 \bm{S}_2 \bm{V}^{T}_{2}$};
\draw[arrow3](mB.east)--(svd2.west) node[below, midway] {decompose} (svd2);
\draw[dashed](svd2.east)--($(svd2.east)+(3.1,0)$);

\node[right=3.8cm of mC] (svd1) {$\bm{U}_1 \bm{S}_1 \bm{V}^{T}_{1}$};
\draw[arrow3](mC.east)--(svd1.west) node[below, midway] {decompose} (svd1);
\draw[dashed](svd1.east)--($(svd1.east)+(4.4,0)$);

% \draw[arrow3]($(mC.south east)+(0.2,0)$)--($(mC.south east)+(0.2,1.94)$);

\end{tikzpicture}

\caption[Singular Value Decomposition]{Singular Value Decomposition of the difference maps $p$, for each length scale $r$. The features, i.e. the voxels, are organized in rows. As such, $\bm{U}$ contains the right singular vectors, i.e. the principal directions/axes, $\bm{V}^T$ contains the left singular vectors. Consequently, $\bm{SV}^{T}$ are the principal components.}
\label{fig:svd}
\end{figure}


As most of the variance is usually explained within the first 20 to 30 directions, it is usually not useful to save all of the directions. Use \code{Pca\_maxEigs} to select the number of directions you want to save. The goal now is to select the principal directions that are going to be used to reproject the data onto. This entirely relies on the user, so the principal directions are reshaped and a few files are saved to help:
\begin{itemize}
    % C = U * (S^2 / n-1) * U^T
    \item \textbf{Variance map}: The covariance matrix of $\bm{X}$ is saved as \code{*\_varianceMap<n>-STD-<r>.mrc}, where \code{<n>} corresponds to the \code{Pca\_maxEigs} and \code{<r>} corresponds to the length scale number $r$ in \code{pcaScaleSpace}. These may be opened on top of your averages from this cycle and should highlight regions where there is significant variability across the data set. If these maps show little blobs everywhere, there is either no significant variability (just picking up noise) or there is still substantial wedge bias at this resolution.
    
    \item \textbf{Principal axes/directions}: The principal axes $\bm{U}$ are saved in \code{*\_eigenImage<n>-STD-mont-<r>.mrc}. The axes are organized from the lower left moving across the bottom row incrementing by one. The first image is associated with the greatest singular value, i.e. it describes the greatest portion of variance in the dataset. The second image describes the greatest portion of remaining variance, and so on. The value of the voxels can be greater than 1 or less than zero, so they should not be interpreted as a grayscale image. It might be easier to visualize them with colors (3dmod: \code{F12}). For visualization, the principal axes are centered and standardized.
    
    \item \textbf{Sums}: The principal axes can be difficult to interpret, so it might useful to add them to the reference, which highlights what is being explained in each principal direction. These are saved in saved in \code{*\_eigenImage<n>-STD-SUM-mont-<r>.mrc}.
    % Actually we calculate the avg.
\end{itemize}

\subsubsection{Clustering}


Once that the principal directions of each length scale $r$ are selected by the user, we can reprojected the data along these axes. We refer to the principal components as ${[\bm{S}_{r}\bm{V}^{T}_r]}_{best}$. Before PCA, each particle $i$ was described by $v$ variables. Now, each particle is only described by $a+b+c$ variables (Figure \ref{fig:cluster}), with $a+b+c \ll v$, i.e. the dimentionality of the dataset was reduced.

\begin{figure}[!htb]  % Stay within section
\captionsetup{labelfont=bf}
\centering

\begin{tikzpicture}[every node/.style={minimum width=1.4cm,minimum height=7mm}]

\node (pcr) {${[\bm{S}_r \bm{V}^{T}_{r}]}_{best}$};
\draw[dashed](pcr.west)--($(pcr.west)+(-3.1,0)$);

\node (pc2) at ($(pcr)+(-1.3,-1.94)$) {${[\bm{S}_2 \bm{V}^{T}_{2}]}_{best}$};
\draw[dashed](pc2.west)--($(pc2.west)+(-1.8,0)$);

\node (pc1) at ($(pc2)+(-1.3,-1.94)$) {${[\bm{S}_1 \bm{V}^{T}_{1}]}_{best}$};
\draw[dashed](pc1.west)--($(pc1.west)+(-0.5,0)$);

% pc all
\matrix (pcall) [draw,matrix of math nodes,fill=white] at ($(pc1)+(+4,-4)$)
{
c_{1,1} & \cdots & c_{1,a+1} & \cdots & c_{1,b+1} & \cdots\\
c_{2,1} & \cdots & c_{2,a+1} & \cdots & c_{2,b+1} & \cdots\\
\cdots & \ddots & \cdots & \ddots & \cdots & \ddots\\
c_{p,1} & \cdots & c_{p,a+1} & \cdots & c_{p,b+1} & \cdots\\
};

\draw[arrow](pcall.south west)--(pcall.south east) node[midway,sloped,below] {$a+b+c$ axes};
\draw[arrow](pcall.north west)--(pcall.south west) node[midway,sloped,below] {$p$ particles};

\node (dima) at ($(pcall)+(-2.75,0)$) [draw,minimum width=2.5cm,minimum height=3.1cm] {};
\draw[arrow](pc1.east)-|($(dima.north)+(0,0)$) node[near end,sloped,above] {$a$ axes};

\node (dimb) at ($(pcall)+(0,0)$) [draw,minimum width=2.5cm,minimum height=3.1cm] {};
\draw[arrow](pc2.east)-|($(dimb.north)+(0,0)$) node[near end,sloped,above] {$b$ axes};

\node (dimc) at ($(pcall)+(2.75,0)$) [draw,minimum width=2.5cm,minimum height=3.1cm] {};
\draw[arrow](pcr.east)-|($(dimc.north)+(0,0)$) node[near end,sloped,above] {$c$ axes};


% kmeans
\matrix (kmeans) [draw,matrix of math nodes,fill=white] at ($(pcall)+(+8,0)$)
{
1 & c(1)\\
2 & c(2)\\
\vdots & \vdots\\
p & c(p)\\
};

\draw[arrow](pcall.east)--(kmeans.west) node[midway,sloped,above] {$k$-means};

\node (tittle_particle) at ($(kmeans)+(-0.75,2.5)$) [rotate=90] {particles};
\node (tittle_class) at ($(kmeans)+(0.64,2.19)$) [rotate=90] {class};


\end{tikzpicture}

\caption[Clustering]{The data is projected onto the principal axes, for each $r$ length scale and stacked into one single $\bm{SV}^T$ matrix of principal components. For visualization, the principal components are transposed to have the axes as columns and the particles as rows. The projected data is then clustered, usually with $k$-means. As a result, each particle is assigned to a class.}
\label{fig:cluster}
\end{figure}


The projected data ${[\bm{SV}^T]}_{best}$ is then clustered into $k$ clusters using $k$-means clustering. $k$ is set by \code{Pca\_clusters}. The squared Euclidean distance metric is used by default, i.e. each centroid is the mean of the points in that cluster, and the number of replicates is set to 128, i.e. the number of times to repeat clustering using new initial cluster centroid positions. Both of these can be changed using the parameters \code{Pca\_distMeasure} and \code{Pca\_nReplicates}. The $k$-means\texttt{++} algorithm is used for cluster center initialization and the number of maximal iteration is set to 5000.

At the end, each particle $p$ is assigned to a cluster $c(p)$, which is saved into the metadata.


%%%%% WORKFLOW


%% PCA:

% 1) extract classification mask info: radius, type, center.

% 2) Combine the reference:
%   - a) Extract the rotation matrix and shifts used to align the two references during FSC calculation.
%   - b) load the 2 half-maps, apply the rotation and shifts to the first half map with spline interpolation and add it to the other map.
%   - c) if the sampling for the classification is not equal to the sampling of the alignment, resample the reference with BH_reScale3d.

% 3) Volume mask, volTMP:
%   - a) get the shape mask (type, size, radius, center). If any symmetry, calculate only the asym unit.
%   - b) get the reference mask and multiply it with the shape mask.
%   - c) save this as *_pcaVolMask.mrc.

% 4) For each resolution band, save the following (X is the resolution band number, starting from 1):
%   - masks.volMask.1.X = volTMP;
%   - masks.binary.1.X = (volTMP >= bh_global_binary_pcaMask_threshold)(:); bh_global_binary_pcaMask_threshold = 0.5
%   - masks.binaryApply.1.X = (volTMP >= 0.01);
%   - masks.scaleMask = bandpass(highcut = 400, lowcut = resolution_band * 0.9)

% 5) Create one average for each resolution band:
%   - a) center and standardize the reference with binaryApply and apply the volMask to this volume.
%   - b) apply the scaleMask of this resolution band and center/standardize.
%   - c) center within binary and standardize within binaryApply and apply binaryApply to the filtered volume.
%   - d) mount and save these averages in test_filt.mrc.

% 6) extract the particles:
%   - a) load the SFs. If different size, pad the reference for each resolution accordingly. 
%   - b) The sampling functions are squared, so sqrt them before applying to the reference.
%   - c) define the particles to use. Either random subset or full dataset.
%   - d) For each valid particle (if not valid: -9999):
%       - i) get shifts and rotation. The first halfset is additionally rotated and shifted (from the FSC) to merge both half-set.
%       - ii) Extract the subtomograms, pad if truncated.
%       - iii) rotate and shift. rotate the particle's SF.
%       - iv) For each resolution band:
%               - apply volMask (reference+shape) to the rotated particle.
%               - bandpass filter with scaleMask and center and standardize.
%               - Calculate the difference map (everything is in Fourier space):
%                   - 1: multiply the reference by the particle's SF (which is rotated in the reference frame).
%                   - 2: center and standardize both ref and particle (which are both in freq space at this point).
%                   - 2: subtract the particle to the reference.
%                   - 3: switch back to real space: this is the difference map.
%               - Save the difference map voxel that belong to binary in tempDataMatrix{resolution_band}.
%       - v) this is done on the GPU, but pull back to the CPU from time to time (every PcaGpuPull particle) in dataMatrix.
%       - vi) clean the dataMatrix with not valid particles due to pre-allocation.

% 7) Center the rows:
%   - For each resolution band, each row (same variable but for all sample), center the row.

% 8) For each resolution band, calculate the decomposition:
%   - a) calculate the economy size SVD of dataMatrix.
%   - b) keep the first n eigen vectors.
%   - c) coeffs = S * V'; S(rxr) and V(rxr). Each eigenvector is scaled (rotate and stretch).
%   - d) Calculate the principal components (projection of the data) varianceMap = U * (S^2/n-1) * U' but i'm sure this is what it does.
%   - e) For each n eigenvector:
%       - i) get the eigenvector (which is actually the eigenImage) from U.
%       - ii) replace it within the full volume (so far it was restricted to binary).
%       - iii) center and standardize within binaryApply. This is the eigenImage.
%       - iv) calculate the sum: (eigenImage + reference)/2.
%   - f) mont the eigenImages and sum and save to disk.

% 9) Save the coeffs to metada.

%%%%%%%%%%%% 
%%%%%%%%%%%% CLUSTER
%%%%%%%%%%%% 

\newpage

