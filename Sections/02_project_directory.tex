\section{The project directory} \label{sec:project_directory}

{\emClarity} should be run from the ``project directory'', referred to as \code{<projectDir>}. Every output will be saved in this directory, ignoring the temporary cache set by the \code{fastScratchDisk} parameter (table \ref{tab:system_param}). As we go along, we will present in more detail each sub-directory and their content.

\begin{itemize}
    \item \code{<projectDir>}:
    \begin{itemize}
        \item \textbf{description}: it contains every input file and input directories {\emClarity} needs and every outputs. As most of the {\emClarity} programs are project based, you should run {\emClarity} from this directory.
        \item \textbf{name}: anything you like.
        \item \textbf{created by}: you.
    \end{itemize}

    \item \code{<projectDir>/fixedStacks}:
    \begin{itemize}
        \item \textbf{description}: it contains the raw (\textit{not} aligned) tilt-series (\code{*.fixed}) and the initial tilt-series alignment files (\code{*.xf}, \code{*.tlt} and optionally \code{*.local} and \code{*.erase}). See section \ref{sec:tilt_series_alignment:ETomo} for more details.
        \item \textbf{name}: fixed.
        \item \textbf{created by}: you or {\emClarity} \code{autoAlign} (\ref{sec:tilt_series_alignment:emClarity}).
    \end{itemize}

     \item \code{<projectDir>/fixedStacks/ctf}:
     \begin{itemize}
        
        \item \textbf{description}: created by \code{ctf estimate} (section \ref{sec:defocus_estimate}) and updated after tilt-series refinements by \code{ctf update} (section \ref{sec:tomoCPR}). It contains the radial averages (\code{*\_psRadial1.pdf}) and stretched power spectrum (\code{*\_PS2.mrc}) computed by \code{ctf estimate}, as well as the tilt-series metadata (\code{*\_ctf.tlt}), used throughout the entire workflow and containing in particular the dose-scheme and defocus estimate of each view.
        \item \textbf{name}: fixed.
        \item \textbf{created by}: {\emClarity} \code{ctf estimate}.
    \end{itemize}

     \item \code{<projectDir>/aliStacks}:
     \begin{itemize}
        \item \textbf{description}: created by \code{ctf estimate} (section \ref{sec:defocus_estimate}) and updated after tilt-series refinement by \code{ctf update} (section \ref{sec:tomoCPR}). It contains the aligned, bead-erased tilt-series. These stacks are mostly used by \code{ctf 3d} to compute the tomograms at different binning.
        \item \textbf{name}: fixed.
        \item \textbf{created by}: {\emClarity} \code{ctf estimate}.
    \end{itemize}

    \item \code{<projectDir>/cache}:
     \begin{itemize}
        \item \textbf{description}: created and updated by {\emClarity} when needed, usually during \code{ctf 3d}. Store any stack or reconstruction for the current binning. If a reconstruction (\code{*.rec}) is present at the current binning, \code{ctf 3d} will skip its reconstruction. This also means that if a reconstruction is aborted, you may need to manually delete a partially complete reconstruction. For older versions of {\emClarity} ($<$ 1.5.1 versions), if the sampling functions are already calculated (\code{*.wgt}), {\emClarity} will re-use them.
        \item \textbf{name}: fixed.
        \item \textbf{created by}: {\emClarity}.
        
    \end{itemize}
    \item \code{<projectDir>/convmap}:
     \begin{itemize}
        \item \textbf{description}: When creating a project with \code{init} (section \ref{sec:init}), {\emClarity} will look in this directory to grab outputs from \code{templateSearch} (section \ref{sec:picking}). If you pick your particles with {\emClarity}, the content of this directory is generated by \code{templateSearch}.
        \item \textbf{name}: fixed.
        \item \textbf{created by}: you / {\emClarity}.
    \end{itemize}

    \item \code{<projectDir>/recon}:
     \begin{itemize}
        \item \textbf{description}: It holds the information for each reconstructed sub-region in a given tilt-series (section \ref{sec:subregions}). The \code{*\_recon.coords} files are read into the metadata created by \code{init} and is used whenever a tomogram is made or whenever the coordinates of a sub-region is needed. Any change to these files is ignored after \code{init}.
        \item \textbf{name}: fixed.
        \item \textbf{created by}: \code{recScript2.sh}
    \end{itemize}

    \item \code{<projectDir>/<binX>}:
     \begin{itemize}
        \item \textbf{description}: {\emClarity} does not directly use this directory, but it is used by \code{recScript2.sh} to define the sub-regions boundaries (section \ref{sec:subregions}) and create \code{<projectDir>/recon}.
        \item \textbf{name}: whatever is defined in \code{recScript2.sh}. By default, \code{bin10}.
        \item \textbf{created by}: \code{recScript2.sh}
    \end{itemize}

    \item \code{<projectDir>/FSC}:
     \begin{itemize}
        \item \textbf{description}: created and updated during subtomogram averaging (section \ref{sec:avg}). It contains the spherical and conical FSCs for each cycle (\code{*fsc\_GLD.txt} and \code{*fsc\_GLD.pdf}), as well as the Figure-Of-Merit used for filtering (\code{*cRef\_GLD.pdf}) and the CTF-corrected volume used for FSC calculations (\code{*noFilt\_EVE.mrc} and \code{*noFilt\_ODD.mrc}). For a full description of the FSC related outputs, see section \ref{sec:algo:avg}. If you run \code{emClarity fsc}, the molecular masks are saved in this directory as well.
        \item \textbf{name}: fixed.
        \item \textbf{created by}: {\emClarity} \code{avg}.
    \end{itemize}
    
    \item \code{<projectDir>/alignResume}:
    \begin{itemize}
        \item \textbf{description}: Contains the results of the subtomogram alignments, for each cycle. {\emClarity} will look at this directory before aligning the particles from a given sub-region. If the results for this sub-region, at the current cycle, are already saved, it will skip the alignment. See section \ref{sec:align:run} for more details on why it is useful.
        \item \textbf{name}: fixed.
        \item \textbf{created by}: {\emClarity} \code{alignRaw}
    \end{itemize}
\end{itemize}
